\documentclass[class=article, crop=false]{standalone}
\input{prepath}
\input{\prepath}

\begin{document}

\section{Discussion 1---April 1, 2021}

\subsection*{Complex Plane}

A complex number $z \in \C$ is a pair of real numbers $x$ and $y$ such that $z = x+iy$ where $i^2=-1$. Moreover, $x$ is the \emph{real part} and $y$ is the \emph{imaginary part}. We denote that $\Re(z)$ and $\Im(z)$, respectively. We may also represent them using `polar coordinates' with $z = r e^{i \phi}$ where $r \ceq |z| \ceq \sqrt{z \overline z} = \sqrt{x^2+y^2}$ and $\phi$ is the \emph{argument} of $z$.

Addition: use the paralleogram law.

Multiplication: sum the angles and multiply the magnitudes.
\begin{figure}[ht]
  \center
  \resizebox{\textwidth}{!}{\figpath{dis1-pic3.pdf_tex}}
  \caption{Multiplication geometrically.}
  % \label{fig:[label]}
\end{figure}


\begin{understandingcheck}{example}
  Compute the following:
    \begin{enumerate}[(a)]
      \item $(1+i)^6$: Notice that $1+i = \sqrt 2 e^{i\pi/4}$, so
        \begin{align*}
          (1+i)^6 &= (\sqrt 2 e^{i\pi/4})^6 \\
            &= 2^3 e^{i3\pi/2} \\
            &= -8i
        \end{align*}
      (We simplify this quickly by seeing this geometrically as going $270^\circ$ around the circle counterclockwise.)

      \item $z$ such that $z^3 = 8i$: Recall that we can write $z$ as $r e^{i\phi}$ for $r > 0$ and $\phi \in \R$. Notice that
        \[
          8i = e^{i \bigpar{\frac{\pi}{2} + 2\pi n}} = z^3 = (e^{i \phi})^3 = r^3 e^{i3 \phi}.
        \]
      Hence, $r^3 = 8$, and $3 \phi = \frac{\pi}{2} + 2\pi n$ for any $n \in \Z$. Then we get $r=2$ and $\phi = \frac{\pi}{6} + \frac{2 \pi }{3}n$ for $n \in \Z$. However, this does not mean we have infinitely many solutions! Because, we just get redundant angles, so we only count the ones between $0$ and $2\pi$. So we have that:
        \[
          z = 2e^{i\pi/6}, 2^{i5\pi/6},2e^{i9\pi/6}.
        \]
      \begin{figure}[ht]
        \center
        \resizebox{3in}{!}{\figpath{dis1-pic4.pdf_tex}}
        \caption{Solutions plotted.}
        % \label{fig:[label]}
      \end{figure}

    \end{enumerate}
\end{understandingcheck}

\begin{understandingcheck}{example}
  \begin{enumerate}[(a)]
    \item Prove: if $f\colon \C \to \C$ is continuous, then for all open subsets $U$ of $\C$, we have that $f\inv(U)\subseteq C$ is open.
      \begin{pf}
        Recall $f \inv(U) \ceq \set{z \in \C : f(z)\in U}$. Fix $z_0 \in f\inv(U)$. We want to show that there exists an $r > 0$ such that $D_r(z_0) \subseteq f\inv(U)$. (This would show the set is open.) Note that $z_0 \in f\inv(U)$ implies that $f(z_0) \in U$. As $U$ is open, there is a radius $\e > 0$ so that $D_\e(f(z_0)) \subseteq U$. As $f$ is continuous on all of $\C$, it is continuous at $z_0$, so there exists a $\delta > 0$ such that $|f(z)-f(z_0)|<\e$ whenever $|z-z_0| < \delta$.

        \begin{figure}[ht]
          \center
          \resizebox{\textwidth}{!}{\figpath{dis1-pic5.pdf_tex}}
          \caption{Diagram of the proof.}
        \end{figure}


        Therefore: $z \in D_\delta(z_0) \ceq \set{w: |w-z_0|<\delta}$ implies $f(z) \in D_\e(f(z_0)) \ceq\set{w : |w-f(z_0)|<\e} \subseteq U$. This is hte definition of $D_\delta(z_0) \subseteq f\inv(U)$ as desired.
      \end{pf}

    \item Prove: if $f\colon \C\to\C$, $\Omega \subseteq \C$ connected, then $f(\Omega)$ is connected.
      \begin{pf}
        Claim: $\clos{f\inv(V)} \subseteq f\inv(\clos V)$ for all $V \subseteq \C$. Fix $V \subseteq \C$. Recall that $\clos V$ is closed (by definition), so $\clos V^c$ is open. Thus, by part (a), we have that $f\inv(\clos V^c) = (f\inv(\clos V))^c$ is open. Thus, $f\inv(\clos V)$ is closed. So
          \[
            \clos{f\inv(V)}\subseteq \clos{f\inv(\clos V)} = f\inv(\clos V)
          \]
        since $f\inv(V) \subseteq f\inv(\clos V)$ and $f\inv(\clos V)$ is closed.

        Suppose that $f(\Omega)$ is not connected. Then there exists $V_1, V_2 \subseteq \C$ such that $f(\Omega) = V_1 \cup V_2$, $\clos V_1 \cap V_2 = \es$, and $V_1 \cap \clos V_2 = \es$. Want: $V_1 = \es$ or $V_2 = \es$. (This is a direct proof since we don't assume that they are nonempty.) Note: $\Omega = f\inv(V_1) \cup f\inv(V_2)$, and
          \[
            \clos{f\inv(V_1)} \cap f\inv(V_2) \subseteq f\inv(\clos V_1) \cap f\inv(V_2)= f\inv(\clos V_1\cap V_2) = f\inv(\es) = \es,
          \]
        with the first containment by the claim and the last line of reasoning from the separatedness of $V_1$ and $V_2$. Similarly,
          \[
            f\inv(V_1) \cap \clos{f\inv(V_2)} \subseteq f\inv(V_1) \cap f\inv(\clos V_2) = f\inv(V_1 \cap \clos V_2) = f\inv(\es) = \es.
          \]
         But $\Omega$ is connected, so $f\inv(V_1) = \es$ or $f\inv(V_2) = \es$, hence $V_1 = \es$ or $V_2 = \es$.
      \end{pf}
  \end{enumerate}
\end{understandingcheck}

\subsection*{Complex derivatives}

$f(z)$ is $\C$-differentiable at $z_0$ $\iff$
  \[
    \lim_{h\to 0}\frac{f(z_0+h)-f(z_0)}{h}
  \]
exists.
\begin{rem}
  Note $h \to 0$ in $\C$, not $\R$.

  \begin{rem}
    (lmao remark in a remark) In $\R$, for limits we come from either left or right, but in $\C$ we can come from up or down, or left or down, or spiral into the point. We can come in from \textbf{any} direction.
  \end{rem}
\end{rem}

\begin{result}{fact}
  $f = u+iv$ is $\C$-differentiable at $x_0 + iy_0$ implies that $F(x,y) = (u(x,y),v(x,y))$ is $\R$-differentiable at $(x_0,y_0)$ and the Cauchy Riemann equations hold at that point.
\end{result}
\begin{rem}
  Cauchy Riemann equatiosn are
    \[
      \frac{\partial u}{\partial x} = \frac{\partial v}{\partial y}, \quad \frac{\partial u}{\partial y} = - \frac{\partial v}{\partial x}.
    \]
\end{rem}

\begin{understandingcheck}{example}
  Find all holomorphic functions $f\colon \C \to \C$ of the form $f(x+iy) = u(y) + i v(x)$.
    \begin{pf}
      $f\colon \C \to \C$ holomorphic implies that $u,v\colon \R^2 \to \R$, $\R$-differentiable and Cauchy Riemann equatiosn hold:
        \[
          \frac{\partial u}{\partial x} = \frac{\partial v}{\partial y}, \quad \frac{\partial u}{\partial y} = - \frac{\partial v}{\partial x}.
        \]
      The first equations becomes $0=0$ since $u$ is only a function of $y$ and $v$ is only a function of $x$. The second equation yields that $u'(y) = -v'(x)$. This tells us that $u'(y)\equiv a$, $v'(x) \equiv -a$ for some $a \in \R$. Integrating, we get that
        \[
          u(y) = ay+b, \quad v(x)=-ax+c
        \]
      for $a,b,c \in \R$. In other words,
        \[
          f(x+iy) = u+iv = (ay+b)+i(-ax+c) = -\underbrace{ia}_{\text{any } i \lam \in i \R}(\underbrace{x+iy}_{z})+(\underbrace{b+ic}_{\text{any } z_0 \in \C}).
        \]
      Conclude: $f(x+iy)= u(y) + iv(x)$ entire $\iff$ $f(z) = i\lam z + z_0$ for $\lam \in \R$ and $z_0 \in \C$.

      (Note: ($\Leftarrow$) implication holds since polynomials are entire.)
    \end{pf}
\end{understandingcheck}

\begin{understandingcheck}{example}
  Prove: if $F = (u,v) \colon \R^2 \to \R^2$ is $\R$-differentiable, then $f(x+iy) = u(x,y) + iv(x,y) \colon \C \to \C$ satisfies
    \[
      f(z) = f(z_0) + \frac{\partial f}{\partial z}(z_0) \cdot (z-z_0) + \frac{\partial f}{\partial \overline z}(z_0) \cdot \overline{(z-z_0)} + \oo(|z-z_0|).
    \]
  Conclude: if $f$ is $\C$-differentiable at $z_0$, then
    \[
      f(z) = f(z_0)+ \frac{\partial f}{\partial z}(z_0) \cdot (z-z_0) + \oo(|z-z_0|) = f(z_0)+ f'(z_0) \cdot (z-z_0) + \oo(|z-z_0|)
    \]
\end{understandingcheck}

\begin{rem}
  $\oo$ means: $g\colon \R^m \to \R^n$ is $\oo(\vbf(x))$ as $x \to 0$ $\iff$
    \[
      \lim_{\vbf x\to 0} \frac{|g(\vbf(x))|}{|\vbf x}=0.
    \]

  Note that:
    \[
      f'(z_0) = \lim_{h \to 0} \frac{f(z_0+h)-f(z_0)}{h} \iff f'(z_0) +\oo(1) = \frac{f(z_0+h)-f(z_0)}{h} \iff f(z_0+h)=f(z_0)+f'(z_0) \cdot h + \oo(h).
    \]

\end{rem}


\end{document}
