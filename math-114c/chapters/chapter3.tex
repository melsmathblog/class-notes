\documentclass[class=article, crop=false]{standalone}
\input{preamble}

\begin{document}

\section{Computably enumerable sets}

\subsection{Partially decidable relations}

In this section, we're going to use the Kleene T predicate a lot, so we're going to restate it here.

\begin{result}{theorem}[Kleene T predicate]
  For each positive integer $n$, there is a recursive $(n+2)$-ary relation $T_n(e, \vec x,y)$ and a computable totla function $U\colon \N \to \N$ which satisfy the following condition: for every $e \in \N$ and every $\vec x \in \N^n$,
    \[
      \varphi_e^{(n)} (\vec x) = U(\mu y[T_n(e, \vec x, y)]).
    \]
\end{result}
\begin{rem}
  Let recall what the Kleene T predicate does: it takes a program code $e$, an input for the program $\vec x$, and a computation history $y$, and decides whether $y$ is a computation history for the program defined by $e$ with input $\vec x$ that halts. It basically simulates a register machine which runs a program with code $e$ and input $\vec x$. Recall that the Kleene T predicate is indeed decidable.
\end{rem}

\begin{defn}
  An $n$-ary relation $R \subseteq \N^n$ is \deft{partially decidable} if and only if the partial function $f\colon \N^n \pto \N$ defined by
    \[
      f(\vec x) =
        \begin{cases}
          1 &\text{if } R(\vec x) \text{ holds}, \\
          \ua &\text{otherwise},
        \end{cases}
    \]
  is computable.

  The function $f$ above is called the \deft{partial characteristic function} of $R$ (which is often denoted $\chi_R^p$) and any algorithm computing $f$ is called a \deft{partial decision procedure} for $R$.
\end{defn}
\begin{rem}
  Partially decidable has other names too; semi-decidable, semi-computable, and partially solvable are other common names. We will try to use partially decidable mainly though. I don't like the other names lmao.
\end{rem}

Do we have any examples of partially decidable relations? We do! In fact, we have many, many such relations. Here's a canonical one, though.

\begin{ex}{\label{ex:unary-diagonal-halting-is-partially-decidable}}
  The (unary diagonal) halting problem (i.e., whether $\varphi_x(x)$ halts) is partially decidable.

  \begin{pf}
    We can define its partial characteristic function $f \colon \N \pto \N$ as the following:
      \[
        f(x) = 1 + 0 \cdot \mu y [T(x,x,y)].
      \]
    Observe that $f$ is computable since it is the minimalization of a decidable relation. Then, notice that if $\varphi_x(x) \da$, then there does exist a halting computation history $y$ and our minimalization will eventually return a value, so $f(x) = 1$ if $\varphi_x(x) \da$. If $\varphi_x(x) \ua$, then there does not exist a halting computation history $y$ and our minimalization never ends, so $f(x) \ua$.
  \end{pf}
\end{ex}
\begin{rem}
  This trick of multiplying a minimalization by $0$ to encode divergence of a computation is SUPER useful. Be aware of its existence!
\end{rem}

\begin{ex}{\label{ex:decidable-implies-partially-decidable}}
  Any decidable relation is partially decidable.

  \begin{pf}
    Suppose $R \subseteq \N^n$ is decidable. Then, $f \colon \N^n \to \N$ defined by
      \[
        f(\vec x) = 1 + 0\cdot \mu y[R(\vec x) \land y=y].
      \]
    If $R(\vec x)$ holds, then the minimalization ends and we get $f(\vec x) = 1$ as desired. IF $R( \vec x)$ does not hold, then our minimalization never ends and the computation diverges, so $f(\vec x) \ua$.
  \end{pf}
\end{ex}
\begin{rem}
  I told you the multiplying by $0$ trick would be useful!
\end{rem}

We know that there exist decidable relations, but do there exist not partially decidable relations? Yes, there do! We actually have another `simple' example: the negation of the (unary diagonal) Halting Problem.
\begin{ex}
  Let $Q(x)$ hold if and only if $\varphi_x(x) \ua$. The relation $Q$ is not partially decidable.

  \begin{pf}
    Suppose towards a contradiction that $Q$ is partially decidable. In particular, suppose that the the partial characteristic function of $Q$, $f_Q \colon \N \pto \N$, is computable. Morever, let $f_H \colon \N \pto \N$ be the (computable) partial characteristic function of the (unary diagonal) Halting Problem. Now, by running $f_Q(x)$ and $f_H(x)$ at the same time, we may decide whether $\varphi_x(x)$ halts (as we will be able to decide if it does or if it does not based on the ouput of $f_H(x)$ and $f_Q(x)$, respectively). Since the Halting Problem is undecidable we have a contradiction and thus $Q$ must not have been partially decidable.
  \end{pf}
\end{ex}

The next two theorems will provide some neat equivalent characterizations of partially decidable relations.

\begin{result}{theorem}{\label{thm:partially-decidable-iff-domain-of-partial-function}}
  An $n$-nary relation $R \subseteq \N^n$ is partially decidable if and only if $R$ is the domain of a computable partial function, i.e., there is a computable partial function $g\colon \N^n \pto \N$ such that for all $\vec x \in \N^n$,
    \[
      R(\vec x) \iff g(\vec x) \da.
    \]
\end{result}
\begin{pf}
  ($\Rightarrow$) Suppose $R$ is partially decidable. Then the domain of $R$'s partial characteristic function clearly equals $R$.

  ($\Leftarrow$) Suppose $R = \Dom(\varphi_e^{(n)})$. Then the partial characteristic function of $R$, $\chi_R^p(\vec x) = 1 + 0 \cdot \mu y[T_n(e,\vec x,y)]$ is computable.
\end{pf}
\begin{rem}
  It follows that $W_0^{(n)}, W_1^{(n)}, W_2^{(n)}, \ldots$ is an (effective) enumeration of all partially deicdable $n$-ary relations.
\end{rem}

\begin{result}{theorem}
  An $n$-ary relation $R \subseteq \N^n$ is partially decidable if and only if there is a $(n+1)$-ary decidable relation $Q \subseteq \N^{n+1}$ such that for any $\vec x \in \N^n$,
    \[
      R(\vec x) \iff \exists y [Q(\vec x,y)].
    \]
\end{result}
\begin{pf}
  ($\Rightarrow$) Suppose $R$ is partially decidable. Then $R = W_e^{(n)}$ for some code $e \in \N$. We then have that
    \[
      R(\vec x) = \varphi_e^{(n)}(\vec x) \da \iff \exists y [T_n(e,\vec x,y)]
    \]
  where $T_n$ is the Kleene T predicate. (To be more explicit, deifne $Q(\vec x,y)$ to hold if and only if $T_n(e, \vec x, y)$.)

  ($\Leftarrow$) Suppose $R$ is a rleation satisfying
    \[
      R(\vec x) \iff \exists y[Q(\vec x,y)]
    \]
  for a decidable $(n+1)$-ary relation $Q \subseteq \N^{n+1}$. Define $f(\vec x) = \mu y[Q(\vec x,y)]$. Then $f$ is computable and $R = \Dom(f)$, so $R$ is partially decidable.
\end{pf}

The next theorem gives closure properties of the class of partially decidable relations.

\begin{result}{theorem}
  The class of partially decidable relations is closed under $\land$, $\lor$, $\exists^<$, $\forall^<$, and $\exists$. It is not closed under negation $\lnot$. More formally:
  \begin{enumerate}[(a)]
    \item If $R$ is partially decidable $n$-ary relations, then the relations $(R \land Q$) and $(R \lor Q)$ are also partially decidable $n$-ary relations.

    \item If $R$ is a partially decidable $(n+1)$-ary relation, then the relations
      \begin{align*}
        (\exists^< R)(\vec x,y) \quad &\iff_\df \quad (\exists z < y) [R(\vec x,z)], \\
        (\forall^< R) (\vec x,y) \quad &\iff_\df \quad (\forall z <y)[R(\vec x,z)], \\
        (\exists R)(\vec x) \quad &\iff_\df \quad \exists y [R(\vec x,y)],
      \end{align*}
    are also partially decidable.

    \item There are partially decidable relations $R$ such that $\lnot R$ is not partially decidable.
  \end{enumerate}
\end{result}
\begin{pf}
  This is annoying. Maybe later lmao.
  % We list this proofs in parts for ease of reading.
  %
  % \begin{enumerate}[(a)]
  %   \item ($\land$) Let $R = \Dom(f)$ or $Q = \Dom(g)$ for some computable partial functions $f,g \colon \N^n \pto \N$. Then, $(R \land Q) = \Dom(f+g)$.
  %
  %   ($\lor$) \textbf{How ot prove the closure under or?}
  %
  %   \item ($\exists^<$)
  %
  %   ($\forall^<$)
  %
  %   ($\exists$) Since $R$ is partially decidable, there exists a decidable relations $Q$ such that $R(\vec x,y)$ holds if and only if $\exists z [Q(\vec x,y,z)]$. Then,
  %     \begin{align*}
  %       (\exists^< R) (\vec x,y) &\iff (\exists z<y)[R(\vec x,z)] \\
  %         &\iff (\exists z < y)(\exists w)[Q(\vec x,z,w)] \\
  %         &\iff (\exists a)[(a)_0 < y \land Q(\vec x, (a)_0, (a)_1)]
  %     \end{align*}
  %   which is partially decidable since projection, comparison, and $Q$ are decidable/computable. T
  % \end{enumerate}
\end{pf}


\begin{result}{corollary}
  If $R$ is an $(n+k)$-ary partially decidable relation, then the $n$-ary relation
    \[
      Q(\vec x) \quad \iff_\df \quad (\exists y_1)\cdots(\exists y_k)[R(\vec x,y_1, \ldots, y_k)]
    \]
  is also partially decidable.
\end{result}
\begin{rem}
  We can rewrite our bunch of existentials as a single number being projected many times, like $Q(\vec x) \iff (\exists a)[R(\vec x, (a)_0, \ldots, (a)_{k-1})]$.
\end{rem}

Here are some more partially decidable relations using these properties.
\begin{ex}
  The following relations are partially decidable for each fixed $n$.
  \begin{enumerate}[(1)]
    \item $R^{(n)}(e) \quad \iff_\df \quad W_e^{(n)} \neq \es$.
      \begin{pf}
        We may rewrite our relation as:
          \[
            W_e^{(n)}\neq \es \iff (\exists x_0) \cdots (\exists x_{n-1})[\varphi_e^{(n)}(\vec x) \da],
          \]
        This is partially decidable since partialy decidability is closed under many existentials and $\varphi_e^{(n)}(\vec x) \da$ is partially decidable. Thus $R^{(n)}$ is partially decidable.
      \end{pf}

    \item $Q^{(n)}(e,y) \quad \iff_\df \quad y \in E_e^{(n)}$.
    \begin{pf}
       Same many existential trick again.
    \end{pf}
  \end{enumerate}
\end{ex}

Now here comes a way to relate decidable and partially decidable relations.
\begin{result}{theorem}
  Let $R$ be an $n$-ary relation. The relation $R$ is decidable if and only if $R$ and $\lnot R$ are both partially decidable.
\end{result}
\begin{pf}
  ($\Rightarrow$) Since $R$ is decidable, $\lnot R$ is also decidable. Then by Example \ref{ex:decidable-implies-partially-decidable}, $R$ and $\lnot R$ are partially decidable.

  ($\Leftarrow$) Suppose $R$ and $\lnot R$ are partially decidable. Let $f,g \colon \N \pto \N$ be such that
    \[
      R(\vec x) \iff f(\vec x)\da \quad \text{and} \quad \lnot R(\vec x) \iff g(\vec x) \da.
    \]
  Notice that for any $\vec x$, $R(\vec x)$ holds (and hence $f(\vec x) \da$) or $\lnot R(\vec x)$ holds (and hence $g(\vec x)\da$), but not both! Then, running both $f(\vec x)$ and $g(\vec x)$ and seeing which halts is a decision procedure for $R$ since if $f(\vec x)$ halts, $R$ holds, and if $g(\vec x)$ halts, $R$ does not hold.
\end{pf}

\begin{result}{corollary}
  Any relation $R$ which is partially decidable but not decidable has $\lnot R$ not partially decidable.
\end{result}

\subsubsection{Graphs of computable partial functions}

Now let's talk about graphs! When we talked about total functions, we saw that a total function was computable if and only if its graph was decidable. It turns out that we actually have an analagous result for partial functions. Indeed, it is:
\begin{result}{theorem}
  Let $f\colon \N^n \to \N$ be a partial function. Then, $f$ is computable if and only if its graph relation $G_f$ is partially decidable.
\end{result}
\begin{pf}
  ($\Rightarrow$) Suppose $f$ is computable and has a code $e \in \N$. Then,
    \[
      f(\vec x) \da  = y \iff \exists z [T_n(e,\vec x,z) \land U(z) = y]
    \]
  which is partially decidable since the statement being quantified over is indeed decidable.

  ($\Leftarrow$) Suppose $G_f$ is partially decidable. Pick a decidable relation $R$ such that
    \[
      G_f(\vec x,y) \iff \exists z [R(\vec x,y,z)].
    \]
  We may then define $f$ by
    \[
      f(\vec x) = (\mu a[R(\vec x, (a)_0, (a)_1)])_0.
    \]
  Thus $f$ is computable since it is defined through minimalization with a decidable relation and substitution with computable functions.
\end{pf}











\subsection{Recursively enumerable sets}

Before we start we're going to do some quick review/comments.
\begin{itemize}
  \item Unary relations on $\N$ are the same things as subsets of $\N$.
  \item Another name for decidable unary relations on $\N$ are called \deft{computable} or \deft{recursive} sets.
  \item A subset $A$ of $\N$ is decidable/computable/recursive if and only if its characteristic function is computable.
  \item A subset $A$ of $\N$ is partially-decidable/semi-computable/semi-recursive if and only if its partial characteristic function is computable.
  \item All of the theorems of the previous section about partially decidable relations apply to these partially decidable sets (unary relations). For example: (1) if $A$ and $B$ are partially decidable subsets of $\N$, then $A \cup B$ and $A \cap B$ are partially decidable (since partial decidability is closed under $\land$ and $\lor$); and (2) a subset $A$ of $\N$ is decidable if and only if $A$ and $\N \sm A$ are partially decidable.
\end{itemize}
\begin{rem}
  That last set of names are kind of garbage imo, so we're going to stick to partially decidable (until we learn about a shorter name).
\end{rem}

Here's an example using an something we know but with this new set up.
\begin{ex}
  The unary, diagonal Halting Problem as a set is usually denoted by $K \ceq \set{x \in \N : \varphi_x(x) \da}$. We know from Example \ref{ex:unary-diagonal-halting-is-partially-decidable} that $K$ is partially decidable.
\end{ex}

Here's the name we were alluding to earlier.
\begin{defn}
  A set $A \subseteq \N$ is \deft{recursively enumerable} (or \deft{computably enumerable}) if and only if either $A$ is empty or there exists a computable total function $f\colon \N \to \N$ such that $A = \Ran(f)$.
\end{defn}
\begin{rem}
  We say $f$ enumerates $A$ since
    \[
      A = \set{f(0), f(1), f(2), \ldots}.
    \]
  Notice that our list above may have repeats.
\end{rem}
\begin{rem}
  We often abbreviate recursively enumerable by r.e. (which we will do from now on).
\end{rem}

Now here's why we said we may be able to use this new name instead of partially decidable.
\begin{result}{theorem}
  A set $A \subseteq \N$ is partially decidable if and only if it is r.e.
\end{result}
\begin{pf}
  ($\Leftarrow$) Suppose $A$ is r.e. Then there is a computable total function $f \colon \N \to \N$ such that $A = \Ran(f)$. Then,
    \[
      y \in A \iff \exists x [f(x)=y].
    \]
  Since $f$ is computable total, its graph is decidable, so $A$ is partially decidable.

  ($\Rightarrow$) Suppose $A$ is partially decidable. If $A$ is empty or finite, it is clearly r.e., so we may assume that $A$ is infinite. Then there exists a decidable relation $R \subseteq \N^2$ such that
    \[
      x \in A \iff \exists y [R(x,y)].
    \]
  Now define $f \colon \N \to \N$ by
    \[
      f(n) = (\mu z [ z \geq n \land R((z)_0,(z)_1)])_0.
    \]
  If $x \in A$, let $z = \cyc{x,y}$ such that $R(x,y)$. Then $f(z) = x$. \textbf{Proof read}
\end{pf}

Here's basically a list of all our equivalent definitions of r.e. that we have established.
\begin{result}{theorem}
  For a set $A \subseteq \N$, the following are equivalent:
  \begin{enumerate}[(i)]
    \item $A$ is r.e.
    \item $A$ is partially decidable (i.e., the partial characteristic function of $A$ is computable).
    \item $A = W_e$ for some $e \in \N$.
    \item There is a decidable binary relation $R$ such that for every $x \in \N$,
      \[
        x \in A \iff \exists y [R(x,y)].
      \]
    \item There is a partially decidable $(n+1)$-ary relation $R$ such that for every $x \in \N$,
      \[
        x \in A \iff \exists y_1 \cdots \exists y_n [R(x,y_1, \ldots, y_n)].
      \]
    \item $A$ is finite or $A$ is the range of a computable injective total function.
    \item $A = E_e$ for some $e \in \N$.
  \end{enumerate}
\end{result}

\begin{ex}{\label{ex:being-not-total-is-not-r.e.}}
  The set $\set{x \in \N : \varphi_x \text{ is not total}}$ is not r.e.

  \begin{pf}
    We will computably reduce membership in $\clos K \ceq \N \sm K$ to membership in $A \ceq \set{x \in \N : \varphi_x \text{ is not total}}$. (i.e., we want a computable total $f\colon \N \to \N$ such that $x \in \clos K \iff f(x) \in A$.)

    Define
      \[
        g(x,y) =
          \begin{cases}
            0 & \text{if } \varphi_x(x)\da, \\
            \ua & \text{otherwise}.
          \end{cases}
      \]
    Since $\varphi_x(x)\da$ is partially decidable, $g$ is computable. By the s-m-n Theorem, there is a computable total function such that
      \[
        \varphi_{f(x)}(y) = g(x,y)
      \]
    for all $x,y \in \N$. Then we have that
      \[
      x \in \clos K \iff \varphi_x(x) \ua\\
        \iff \forall y [g(x,y) \ua] \\
        \iff W_{f(x)} = \es \\
        \iff \varphi_{f(x)} \text{ is not total}.
      \]
    (Note that the last equivalence holds only because of the way we defined $g$ and hence $f$. It acts sort of like a `all or nothing' relation where it's either empty or all of $\N$.) Thus if $A$ was r.e., then $\clos K$ would be r.e. which is absurd! Thus $A$ is not r.e.
  \end{pf}
\end{ex}
\begin{rem}
  For a subset $A$ of $\N$, we denote its complement by $\clos A$ (i.e., $\clos A \ceq \N \sm A$).
\end{rem}

\begin{result}{theorem}{\label{thm:increasing-enumeration-is-recursive}}
  Let $A \subseteq \N$ be an infinite set. $A$ is recursive if and only if $A$ is the range of a computable total function which is strictly increasing (i.e., $A = \set{f(0) < f(1) < f(2) < \cdots}$).
\end{result}
\begin{pf}
  ($\Rightarrow$) Suppose $A$ is infinite and recursive. Define $f\colon \N \to \N$ by recusion as follows: let $w_0$ be the least element of $A$ and define $f$ by
    \[
      \begin{cases}
        f(0) &= w_0, \\
        f(n+1) &= \mu x [x \in A \land x > f(n)].
      \end{cases}
    \]
  Notice that $f$ is then computable since it is defined from recursion by computable functions (in particular, a minimalization of a decidable relation). We know that $f$ enumerates $A$ since find a new element of $A$ with every new value of $f$.

  ($\Leftarrow$) Suppose $f$ is computable and enumerates $A$ in increasing order. Then $f$ is strictly increasing, so $f(n) \geq n$. Then apply our midterm problem.
\end{pf}

\begin{result}{theorem}
  Every infinite r.e. set has an infinite recursive subset.
\end{result}
\begin{pf}
  Let $A$ be an infinite r.e. set. Pick a computable total function $f\colon \N \to \N$ such that $A = \Ran(f)$.

  Define $g\colon \N \to \N$ by
    \[
      \begin{cases}
        g(0) &= f(0) \\
        g(n+1) &= f(\mu x [f(x) > g(n)]).
      \end{cases}
    \]
  Our function $g$ is computable since we are composing computable functions (the inner argument is computable since it is the minimalization of a decidable comparison). Furthermore, $g$ is total since $A$ is infinite. Finally, because $g$ is strictly increasing, $\Ran(g) \subseteq A$ is recursive by Theorem \ref{thm:increasing-enumeration-is-recursive}.
\end{pf}









\subsection{Many-one reducibility}

Notice how we made particular names for unary relations being decidable? We're doing the same exact thing for computable reductions/reducibility; we call computable reductions on unary relations/subsets of $\N$ are called \deft{many-one reductions}.

\begin{defn}
  Let $A$ and $B$ be subsets of $\N$. $A$ is \deft{many-one reducible} (or \deft{$m$-reducible}) to $B$ if and only if there is a computable total function $f\colon \N \to \N$ such that
    \[
      x \in A \iff f(x) \in B.
    \]
  Such an $f$ is called an \deft{$m$-reduction} from $A$ to $B$.

  If $A$ is m-reducible to $B$, we write $A \leq_m B$.

  For a computable total function $f\colon \N \to \N$, we write $f\colon A \leq_m B$ to mean that $f$ is an $m$-reduction from $A$ to $B$.
\end{defn}

\begin{figure}[ht]
  \center
  \resizebox{\textwidth}{!}{\figpath {mone.pdf_tex}}
  \caption{A drawn example of $A \leq_m B$ for $A$ (red) and $B$ (blue) subsets of $\N$.}
  \label{fig:many-one-reducibility}
\end{figure}



\begin{rem}
  We say "many-one" because the reductions aren't required to be injective. When we do require them to be injective, we call it $1$-reducibility where $A$ is $1$-reducible to $B$. But we won't work with this so don't worry about it. $\colon >$
\end{rem}

\begin{ex}
  If $A$ is a nonempty subset of $\N$, then $\N \leq_m A$.

  \begin{pf}
    Since $A$ is nonempty, we can pick some $a \in A$. Define $f\colon \N \to \N$ by $f(x) = a$. Then $f$ is a many-one reduction of $\N$ to $A$. (See the accompanying figure for a visual.)
  \end{pf}

  \begin{center}
    \resizebox{\textwidth}{!}{\figpath {N-to-nonempty-m-reduction.pdf_tex}}
  \end{center}
\end{ex}


\begin{ex}
  $\set{x \in \N : \varphi_x \text{ is total}} \leq_m \set{x \in \N : \varphi_x = \underline 0}$.

  \begin{pf}
    Define $g\colon \N^2 \to \N$ by
      \[
        g(x,y) =
          \begin{cases}
            0 & \text{if } \varphi_x(y) \da, \\
            \ua & \text{otherwise}.
          \end{cases}
      \]
    Notice that $g$ is computable since $\varphi_x(y)\da$ is partially decidable (it is an iteration of the Halting Problem). By s-m-n Theorem, there exists a computable total function $f\colon \N \to \N$ such that $\varphi_{f(x)}(y) = g(x,y)$ for all $x,y$. We then have that
      \[
        \varphi_x \text{ is total} \iff g(x,y) \da=0 \text{ for all } y \iff \varphi_{f(x)} = \underline 0.
      \]
    Thus, $f\colon \set{x \in \N : \varphi_x \text{ is total}} \leq_m \set{x \in \N : \varphi_x = \underline 0}$.
  \end{pf}
\end{ex}

We now recall  here one application of many-one reducibility that we've mentioned before when talking about relations.

\begin{result}{proposition}
  If $A \leq_m B$ and $B$ is recursive, then $A$ is recursive. This is logically equivalent to the statement: if $A\leq_m B$ and $A$ is not recursive, then $B$ is not recursive.
\end{result}

In our homework we proved a similar thing about r.e. sets.
\begin{result}{proposition}{\label{prop:reduce-to-r.e.-is-r.e.}}
  If $A \leq_m B$ and $B$ is r.e., then $A$ is r.e. This is logically equivalent to the statement: if $A\leq_m B$ and $A$ is not r.e., then $B$ is not r.e.
\end{result}

Here's the thing we proved in homework so you have some context.
\begin{result}{theorem}
  If $R \subseteq \N^m$ and $Q \subseteq \N^n$ are relations, $R$ is partially decidable, and $R$ can be computably reduced to $Q$, then $Q$ is not partially decidable.
\end{result}
\begin{pf}
  Suppose that $R$ is an $m$-ary not partially decidable relation that can be computably reduced to an $n$-ary relation $Q$. We aim to show that $Q$ is also not partially decidable. Suppose towards a contradiction that $Q$ is partially decidable. Then its characterstic function, $\chi_Q \colon \N^n \pto \N$, is a computable partial function. Let $f_0, \ldots, f_{n-1} : \N^m \to \N$ be the computable reduction from $R$ to $Q$. Then, $\chi_R(\vec x) \sqeq \chi_Q(f_0(\vec x), \ldots, f_{n-1}(\vec x))$ where $\chi_R$ is the characteristic function for $R$. Then $\chi_R$ is computable partial, but this is absurd since $R$ is not partially decidable. Thus $Q$ must not have been partially decidable. This completes the proof.
\end{pf}

It should be clear that this general theorem about partially decidable sets and computable reductions applies to our subset sof $\N$ (because they are just unary relations).

We saw an example of using Proposition \ref{prop:reduce-to-r.e.-is-r.e.} in Example \ref{ex:being-not-total-is-not-r.e.}.

We turns out that we can almost sort of maybe order sets based on whether they are many-one reducible to each other. We'll put this in a formal statement now.
\begin{result}{proposition}{\label{prop:leq_m-as-a-preorder}}
  $\leq_m$ is a preorder on the power of $\N$, i.e.
    \begin{enumerate}[(i)]
      \item $\leq_m$ is reflexive: $A \leq_m A$ for every $A \subseteq \N$.
      \item $\leq_m$ is transitive: if $A \leq_m B$ and $B\leq_m C$, then $A \leq_m C$.
    \end{enumerate}
\end{result}
\begin{pf}
  (i) The identity function $U_1^1$ is a $m$-reduction from a set $A$ to itself.

  (ii) If $f\colon A \leq_m B$ and $g \colon B \leq_m C$, then $g \circ f$ is computable and for any $x$,
    \[
      x \in A \iff f(x) \in B \iff g(f(x)) \in C.
    \]
  Thus, $g \circ f \colon A \leq_m C$.
\end{pf}
\begin{rem}
  $\leq_m$ is not a partial order because it is not antisymmetric: there exists $A,B \subseteq \N$ such that $A \leq_m B$ and $B \leq_m A$, but $A \neq B$. For example, take the set of even numbers and the set of odd numbers. Then the successor function is an $m$-reduction both ways but the sets are clearly not equal.
\end{rem}

It turns out that some things, in a way, both comparable and incomparable to other r.e. sets. What I mean by this vague comment is the following: there are sets that are \emph{maximally} complicated, i.e., there are r.e. sets that are as complicated as all other r.e. sets. Very cool stuff B). Let's put this more formally.
\begin{defn}
  A set $A$ is \deft{$m$-complete} if and only if it is r.e. and $B \leq_m A$ for all r.e. sets $B$.
\end{defn}

We immediately have an example of such a set: the Halting Problem! Very neat that this canonical example of undecidability comes up all the time is pretty neat imo.

\begin{result}{theorem}
  $K \ceq \set{x \in \N : \varphi_x(x) \da}$ is $m$-complete.
\end{result}
\begin{pf}
  Let $B$ be an r.e. set. Then there exists a recursive $2$-ary relation $R$ such that for all $x \in \N$,
    \[
      x \in B \iff \exists y [R(x,y)].
    \]
  (Note: We want that a computable total function $f$ such that $x \in B \iff \varphi_{f(x)}(f(x))\da \iff f(x) \in K$.)

  Define $g(x,y) = \mu z [R(x,z)]$, so
    \[
      g(x,y) \da \text{ for all } y \in \N \iff x \in B.
    \]
  By s-m-n Theorem, there is a computable $f\colon \N \to \N$ such that $\varphi_{f(x)}(y) = g(x,y)$ for all $x,y$. Thus,
    \[
      x \in B \iff g(x,y) \da \text{ for all } y \in \N \iff g(x,f(x)) \da \iff \varphi_{f(x)}(f(x)) \da \iff f(x) \in K,
    \]
  so $f\colon B \leq_m K$.
\end{pf}

Do we have more examples of $m$-complete sets? We do! But not just yet. The following result should help us find a ton more.
\begin{result}{proposition}
  If $A$ is $m$-complete, $B$ is r.e., and $A \leq_m B$, then $B$ is $m$-complete.
\end{result}
\begin{pf}
  (This is mainly by transitivity of $\leq_m$.) If $B$ is r.e., $A$ is $m$-complete, and $A \leq_m B$, then for any r.e. set $C$, $C \leq_m A \leq_m B$ which then says by transitivity that $C \leq_m B$. Thus $B$ is $m$-complete.
\end{pf}

Thus we have that any time we reduce $K$ to any other r.e. set, that other set is another example of an $m$-complete set. So, here are two immediate examples.
\begin{ex}
  \begin{enumerate}[(i)]
    \item $\set{x \in \N : \varphi_x(x) \da = 0}$ is $m$-complete.
    \item $\set{x \in N : x \in E_x}$ is $m$-complete.
  \end{enumerate}
\end{ex}

Now here's an interesting question that my professor posed: does there exist an r.e. set which is not recursive \emph{and} is not $m$-complete? It turns out that the answer is yes, there is! But we don't have any examples at the moment. But rest assured, (according to lecture) we will prove that such sets do in fact exist.









\subsubsection{Degrees}

\begin{defn}
  Sets $A$ and $B$ are said to be $m$-equivalent, denoted $A \equiv_m B$, if $A \leq_m B$ and $B \leq_m A$.
\end{defn}

\begin{result}{proposition}
  $m$-equivalence, $\equiv_m$, is an equivalence relation on $\mathcal P(\N)$.
\end{result}
\begin{pf}
  This should be pretty clear from the preorder properties of $\leq_m$.
\end{pf}

\begin{defn}
  For a set $A$, we denote its $\equiv_m$-equivalence class by
    \[
      d_m(A) \ceq \set{B \in \mathcal P(\N) : A \equiv_m B}.
    \]
  $d_m(A)$ is called the $m$-degree of $A$.
\end{defn}
\begin{rem}
  Motivation for terminology: if $d_m(A) = d_m(B)$, i.e., $A \equiv_m B$, then $A$ and $B$ have the same \emph{degree} of difficulty with respect to $\leq_m$.
\end{rem}

\begin{ex}
  $\set{x \in N : x \text{ is even}} \equiv_m \set{x \in \N : x \text{ is odd}}$.

  \begin{pf}
    We mentioned this already with the successor function in the remark to Proposition \ref{prop:leq_m-as-a-preorder}.
  \end{pf}
\end{ex}

\begin{ex}
  $K \equiv_m A \ceq \set{x \in \N : \varphi_x(x) \da = 0}$.

  \begin{pf}
    Since $K$ is $m$-complete and $A$ is r.e., we have by definition of $m$-complete that $A \leq_m K$. We also showed in the homework that $A \leq_m K$. \textbf{finish. maybe fix this up and actually include the proof if it's not in the notes already.}
  \end{pf}
\end{ex}

\begin{ex}
  If $A$ and $B$ are nontrivial, i.e., not $\es$ nor $\N$, recursive sets, then $A \equiv_m B$.

  \begin{pf}
    Fix $b \in B$ and $c \in \clos B$. Define $f\colon \N \to \N$ by
      \[
        f(x) =
          \begin{cases}
            b & \text{if } x\in A, \\
            c & \text{if } x \nin A.
          \end{cases}
      \]
    Then $f$ is computable since it is defined by cases with decidable relations and constant functions. Thus, $f\colon A \leq_m B$. It is similar to show $B \leq_m A$.
  \end{pf}
\end{ex}

\begin{nota}
  Often we use boldface letters to denote $m$-degree, $\vec a$, $\vec b$. In handwriting, we underline: \underline a, \underline b, etc.
\end{nota}

Note that for $m$-degrees $\vec a$, $\vec b$,
  \[
    A \leq_m B \text{ for some } A \in \vec a, B \in \vec b \iff A\leq_m B \text{ for all } A \in \vec a, \, B \in \vec b.
  \]

\begin{defn}
  For $m$-degrees $\vec a$, $\vec b$, we write $\vec a \leq_m \vec b$ if and only if $A \leq_m B$ for all (equivalently, some) $A \in \vec a$, $B \in \vec b$.

  We write $\vec a <_m \vec b$ if and only if $\vec a \neq \vec b$ and $a \leq_m \vec b$.
\end{defn}

\begin{result}{proposition}
  $\leq_m$ is a partial ordering of the $m$-degrees, i.e., $\leq_m$ is reflexive, transitive, and antisymmetric on $m$-degrees.
\end{result}
\begin{pf}
  Let $\vec a$, $\vec b$, and $\vec c$ be $m$-degrees.

  (Reflexive) Let $A \in \vec a$. Then, $A \leq_m A$, so $\vec a \leq_m \vec a$.

  (Transitive) Suppose $\vec a \leq_m \vec b \leq_m \vec c$ and let $A \in \vec a$, $B \in \vec b$, and $C \in \vec c$. Then, $A \leq_m B \leq_m C$. Thus we have that $A \leq_m C$, so $\vec a \leq_m \vec c$.

  (Antisymmetric) Suppose $\vec a \leq_m \vec b$ and $\vec b \leq_m \vec a$. Fix $A \in \vec a$ and $B \in \vec b$. Since $\vec a \leq_m \vec b$, $A \leq_m B$. Similarly, since $\vec b \leq_m \vec a$, $B \leq_m A$. Then, $A \equiv_m B$, so
    \[
      \vec a = d_m(A) = d_m(B) = \vec b.
    \]
\end{pf}











\subsubsection{Lay of the land of $m$-degrees}

\begin{nota}
  We denote $\vec o \ceq d_m(\es)$, $\vec n \ceq d_m(\N)$, $\vec 0_m \ceq d_m(A)$, where $A$ is any recursive set other than $\es$ and $\N$, and $\vec 0_m' \ceq d_m(K)$.
\end{nota}

\begin{defn}
  Call a degree an \deft{r.e. $m$-degree} if it contains an r.e. set $A \in \vec a$.
\end{defn}

\begin{ex}
  $\vec 0_m'$ is an r.e. $m$-degree.
\end{ex}

(Cool picture incoming!!) $\vec o, \vec n \leq_m \vec a$ for any $m$-degree $\vec a$, but $\vec o \not \leq_m \vec n$ and $\vec n \not \leq_m \vec o$. We then have the following diagram.

\begin{figure}[ht]
  \center
  \resizebox{\textwidth}{!}{\figpath {lay-of-the-land.pdf_tex}}
  \caption{Diagram of what $m$-degrees look like in a graph.}
  \label{fig:lay-of-the-land}
\end{figure}
\noindent This diagram begs the question: is there an r.e. $m$-degree $\vec a$ which is not any of $\vec o, \vec n, \vec 0_m, \vec 0_m'$? We need to do a lot of work to show that such $m$-degrees do exist, but we'll get there eventually.

I don't have a good intro for this one lmao.
\begin{result}{theorem}
  For any two $m$-degrees $\vec a$, $\vec b$, there is a least upper bound $c$ so that $\vec a \leq_m \vec c$ and $\vec b \leq_m \vec c$, and there is no $\vec d<_m \vec c$ such that $\vec a \leq_m \vec d$ and $\vec b \leq_m \vec d$.
\end{result}
\begin{pf}
  Pick $A \in \vec a$ and $B \in \vec b$. Define
    \[
      C \ceq \set{2x : x \in A} \cup \set{2x+1: x \in B} = A \oplus B.
    \]
  Let $\vec c = d_m(C)$. Note,
    \begin{align*}
      x \in A &\iff 2x \in A \oplus B, \\
      x \in B &\iff 2x+1 \in A \oplus B.
    \end{align*}
  So $f_0 \colon A \leq_m A \oplus B$ and $f_1 \colon B \leq_m A \oplus B$ where $f_0(x) =2x$ and $f_1(x) = 2x+1$, so $\vec c$ is an upper bound for $\vec a$ and $\vec b$.

  Suppose $\vec a, \vec b \leq_m \vec d$. We aimt to show that $\vec c \leq_m\vec d$. Pick $D \in \vec d$ and $h_0 \colon A \leq_m D$, $h_1 \colon B \leq_m D$. Then,
    \begin{align*}
      x = 2k \in A \oplus B &\iff k \in A \iff h_0(k) \in D, \\
      x = 2k+1 \in A \oplus B &\iff k \in B \iff h_1(k) \in D.
    \end{align*}
  Thus, $h\colon A \oplus B \leq_m D$ where
    \[
      h(x) =
        \begin{cases}
          h_0(k) &\text{if } \exists k [x=2k], \\
          h_1(k) &\text{if } \exists k [x=2k+1].
        \end{cases}
    \]
\end{pf}










\subsection{An r.e., non-recursive, non-$m$-complete set}
\textbf{Goal.} We will prove the existence of a set $A$ wihch is r.e., not recursive, and not $m$-complete. In other words, we will find an r.e. $m$-degree $\vec a$ such that
  \[
    \vec 0_m <_m \underbrace{\vec a}_{\ni A} <_m \vec 0_m' = d_m(K).
  \]
In other other words, we're finding a weird seeming set/problem. This goal will take some work and will require us to first study sets which fail to be r.e. in a strong way.

\subsubsection{Productive and creative sets}

If a set $A$ is not r.e., then
  \[
    W_e \subseteq A \implies A \sm W_e \neq \es.
  \]
We study such sets where we can, from an index $e$ with $W_e \subseteq A$, compute a "witness" that $A \sm W_e \neq \es$, i.e.m compute an element $x \in A \sm W_e$. This leads us to our first definition of this little section.

\begin{defn}
  A set $A$ is \deft{productive} if and only if there is a totla computable function $p\colon \N \to \N$ such that for any $e \in \N$,
    \[
      W_e \subseteq A \implies p(e) \in A \sm W_e.
    \]
  The function $p$ is called a \deft{productive function} for $A$.
\end{defn}
\begin{rem}
  Note that if $A$ is productive, then $A$ is not r.e.: if there were $e$ such that $A= W_e$,, then we would get the absurdity that $p(e) \in A \sm W_e = \es$.
\end{rem}

We have an example of such a set, actually! (This kind of surprised me that we had one immediately.)

\begin{result}{proposition}
  $\clos K$ is productive.
\end{result}
\begin{pf}
  Define $p\colon \N \to \N$ by $p(e)=e$. We claim that $p$ is a productive function for $\clos K$.

  Suppose $W_e \subseteq \clos K$. We want to show that $p(e)= e \in \clos K \sm W_e$ (i.e., $e \nin K$ and $e \nin W_e$). Suppose towards a contradiction that $e \in W_e$. Since $W_e \subseteq \clos K$, $e \in \clos K$, so $e \nin K$. Thus, $e \nin W_e$, which is absurd!

  Thus $e \nin W_e$, so $e \nin K$, hence $e \in \clos K \sm W_e$. Thus $p$ is a productive function for $\clos K$ and $\clos K$ is productive.
\end{pf}

We have many more examples! Well, a class of them. I'm not planning to give a long list lmfao.

\begin{result}{proposition}{\label{prop:productive-from-reduction}}
  If $A$ is productive and $A \leq_m B$, then $B$ is productive.
\end{result}
\begin{pf}
  Let $p\colon \N \to \N$ be the productive function for $A$ and let $f\colon A \leq_m B$. First, we show that there is a computable total function $s\colon \N \to \N$ such that
    \[
      W_{s(e)} = f\inv (W_e) \text{ for all } e \in \N.
    \]
  Apply s-m-n Theorem to
    \[
      g(e,x) =
        \begin{cases}
          0 & \text{if } f(x) \in W_e, \\
          \ua & \text{otherwise}.
        \end{cases}
    \]

  Our claim is that the function $q(e) = f(p(s(e)))$ is a productive function for $B$.
    \begin{align*}
      W_e \in B & \stackrel{f\colon A \leq_m B}{\implies} W_{s(e)} = f\inv (W_e) \subseteq A \\
        & \stackrel{p \text{ is productive for } A}{\implies} p(s(e)) \in A \sm W_{s(e)} = A \sm f\inv (W_e) \\
        & \stackrel{f\colon A \leq_m B}{\implies} f(p(s(e))) \in B\sm W_e.
    \end{align*}
\end{pf}

\begin{rem}
  If $K \leq_m A$< then $\clos K \leq_m \clos A$. Thus, for any $m$-complete set $A$, $\clos A$ is not productive. e.g., $A \ceq \set{x \in \N : x \in E_x}$, then $\clos A = \set{x \in \N : x \nin E_x}$ is productive.
\end{rem}

For a set $A$ which is r.e., $A$ is not recursive if and only if its complement is not r.e. The next notion captures the idea of r.e. sets whose complement is not r.e. in a ``strong" way:

\begin{defn}
  A set $A$ is \deft{creative} if it is r.e. and its complement $\clos A$ is productive.
\end{defn}

\begin{ex}{\label{ex:m-complete-implies-creative}}
  $K$ is creative. In fact, any $m$-complete set is creative.

  \begin{pf}
    See the remark for Proposition \ref{prop:productive-from-reduction} to see that $m$-complete implies creative.
  \end{pf}
\end{ex}

\begin{result}{theorem}
  A set $A$ is creative if and only if it is $m$-complete.
\end{result}
\begin{pf}
  \textbf{Ask for a walk through with this proof. FINISH}
  ($\Leftarrow$) This was shown in Example \ref{ex:m-complete-implies-creative}.

  ($\Rightarrow$)

  \begin{rem}
    This is honestly pretty difficult and confusing. It's a clever application of the (amazing) Second Recursion Theorem (which is honestly always pretty weird for me). Because of the difficulty of this argument, we're going to outline much more.
  \end{rem}

  Suppose $A$ is creative. Let $B$ be r.e. We want to show that $B \leq_m A$. Let $p$ be the productive function for $\clos A$.

  \textbf{Claim.} Suffices to find a computable total function $f\colon \N \to \N$ such that
    \[
      W_{f(x)} =
        \begin{cases}
          \set{p(f(x))} & \text{if } x\in B, \\
          \es & \text{otherwise}.
        \end{cases}
    \]

  \textbf{Proof of claim.} Suppose we have such an $f$. Then we show $g(x) = p(f(x))$ is an $m$-reduction from $B$ to $A$.

  \underline{Case 1:} Suppose $x \in B$. Then, $W_{f(x)} = \set{p(f(x))}$ by assumption. Now, suppose towards a contradiction that $p(f(x)) \nin A$. Then $W_{f(x)} \subseteq \clos A$ implies $p(f(x)) \in \clos A \sm W_{f(x)}$, which then yields that $p(f(x)) \neq p(f(x))$ which is absurd!

  \underline{Case 2:} Suppose $x \nin B$. Then, $W_{f(x)} = \es \subseteq \clos A$. Then, $p(f(x)) \in \clos A \sm W_{f(x)} = \clos A$, so $p(f(x)) \nin A$.


  \textbf{Construction of $f$.} Let $s\colon \N^2\to \N$ be such that
    \[
      \varphi_{s(e,x)}(y) = \varphi_e^{(2)}(x,y) \text{ for all } e,x,y \in \N.
    \]
  Define
    \[
      h(e,x,y) =
        \begin{cases}
          0 & \text{if } y = p(s(e,x)) \land x \in B, \\
          \ua & \text{otherwise}.
        \end{cases}
    \]
  By the (amazing) Second Theorem, there is $e^*$ such that
    \[
      \varphi_{s(e,x)}(y) = \varphi_{e^*}(x,y) = h(e^*,x,y).
    \]
  Then define $f(x) = s(e,x)$. Thus we have that
    \[
      \varphi_{f(x)}(y) = \varphi_{s(e^*,x)}(y) =
      \begin{cases}
        0 & \text{if } y = p(s(e,x)) \land x \in B \\
        \ua & \text{otherwise}
      \end{cases} =
      \begin{cases}
        0 & \text{if } y = p(f(x)) \land x \in B \\
        \ua & \text{otherwise}.
      \end{cases}
    \]
\end{pf}








\end{document}
