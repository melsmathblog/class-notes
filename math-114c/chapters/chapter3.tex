\documentclass[class=article, crop=false]{standalone}
\input{preamble}

\begin{document}

\section{Computably enumerable sets}

\subsection{Partially decidable relations}

In this section, we're going to use the Kleene T predicate a lot, so we're going to restate it here.

\begin{result}{theorem}[Kleene T predicate]
  For each positive integer $n$, there is a recursive $(n+2)$-ary relation $T_n(e, \vec x,y)$ and a computable totla function $U\colon \N \to \N$ which satisfy the following condition: for every $e \in \N$ and every $\vec x \in \N^n$,
    \[
      \varphi_e^{(n)} (\vec x) = U(\mu y[T_n(e, \vec x, y)]).
    \]
\end{result}
\begin{rem}
  Let recall what the Kleene T predicate does: it takes a program code $e$, an input for the program $\vec x$, and a computation history $y$, and decides whether $y$ is a computation history for the program defined by $e$ with input $\vec x$ that halts. It basically simulates a register machine which runs a program with code $e$ and input $\vec x$. Recall that the Kleene T predicate is indeed decidable.
\end{rem}

\begin{defn}
  An $n$-ary relation $R \subseteq \N^n$ is \deft{partially decidable} if and only if the partial function $f\colon \N^n \pto \N$ defined by
    \[
      f(\vec x) =
        \begin{cases}
          1 &\text{if } R(\vec x) \text{ holds}, \\
          \ua &\text{otherwise},
        \end{cases}
    \]
  is computable.

  The function $f$ above is called the \deft{partial characteristic function} of $R$ (which is often denoted $\chi_R^p$) and any algorithm computing $f$ is called a \deft{partial decision procedure} for $R$.
\end{defn}
\begin{rem}
  Partially decidable has other names too; semi-decidable, semi-computable, and partially solvable are other common names. We will try to use partially decidable mainly though. I don't like the other names lmao.
\end{rem}

Do we have any examples of partially decidable relations? We do! In fact, we have many, many such relations. Here's a canonical one, though.

\begin{ex}{\label{ex:unary-diagonal-halting-is-partially-decidable}}
  The (unary diagonal) halting problem (i.e., whether $\varphi_x(x)$ halts) is partially decidable.

  \begin{pf}
    We can define its partial characteristic function $f \colon \N \pto \N$ as the following:
      \[
        f(x) = 1 + 0 \cdot \mu y [T(x,x,y)].
      \]
    Observe that $f$ is computable since it is the minimalization of a decidable relation. Then, notice that if $\varphi_x(x) \da$, then there does exist a halting computation history $y$ and our minimalization will eventually return a value, so $f(x) = 1$ if $\varphi_x(x) \da$. If $\varphi_x(x) \ua$, then there does not exist a halting computation history $y$ and our minimalization never ends, so $f(x) \ua$.
  \end{pf}
\end{ex}
\begin{rem}
  This trick of multiplying a minimalization by $0$ to encode divergence of a computation is SUPER useful. Be aware of its existence!
\end{rem}

\begin{ex}{\label{ex:decidable-implies-partially-decidable}}
  Any decidable relation is partially decidable.

  \begin{pf}
    Suppose $R \subseteq \N^n$ is decidable. Then, $f \colon \N^n \to \N$ defined by
      \[
        f(\vec x) = 1 + 0\cdot \mu y[R(\vec x) \land y=y].
      \]
    If $R(\vec x)$ holds, then the minimalization ends and we get $f(\vec x) = 1$ as desired. IF $R( \vec x)$ does not hold, then our minimalization never ends and the computation diverges, so $f(\vec x) \ua$.
  \end{pf}
\end{ex}
\begin{rem}
  I told you the multiplying by $0$ trick would be useful!
\end{rem}

We know that there exist decidable relations, but do there exist not partially decidable relations? Yes, there do! We actually have another `simple' example: the negation of the (unary diagonal) Halting Problem.
\begin{ex}
  Let $Q(x)$ hold if and only if $\varphi_x(x) \ua$. The relation $Q$ is not partially decidable.

  \begin{pf}
    Suppose towards a contradiction that $Q$ is partially decidable. In particular, suppose that the the partial characteristic function of $Q$, $f_Q \colon \N \pto \N$, is computable. Morever, let $f_H \colon \N \pto \N$ be the (computable) partial characteristic function of the (unary diagonal) Halting Problem. Now, by running $f_Q(x)$ and $f_H(x)$ at the same time, we may decide whether $\varphi_x(x)$ halts (as we will be able to decide if it does or if it does not based on the ouput of $f_H(x)$ and $f_Q(x)$, respectively). Since the Halting Problem is undecidable we have a contradiction and thus $Q$ must not have been partially decidable.
  \end{pf}
\end{ex}

The next two theorems will provide some neat equivalent characterizations of partially decidable relations.

\begin{result}{theorem}{\label{thm:partially-decidable-iff-domain-of-partial-function}}
  An $n$-nary relation $R \subseteq \N^n$ is partially decidable if and only if $R$ is the domain of a computable partial function, i.e., there is a computable partial function $g\colon \N^n \pto \N$ such that for all $\vec x \in \N^n$,
    \[
      R(\vec x) \iff g(\vec x) \da.
    \]
\end{result}
\begin{pf}
  ($\Rightarrow$) Suppose $R$ is partially decidable. Then the domain of $R$'s partial characteristic function clearly equals $R$.

  ($\Leftarrow$) Suppose $R = \Dom(\varphi_e^{(n)})$. Then the partial characteristic function of $R$, $\chi_R^p(\vec x) = 1 + 0 \cdot \mu y[T_n(e,\vec x,y)]$ is computable.
\end{pf}
\begin{rem}
  It follows that $W_0^{(n)}, W_1^{(n)}, W_2^{(n)}, \ldots$ is an (effective) enumeration of all partially deicdable $n$-ary relations.
\end{rem}

\begin{result}{theorem}
  An $n$-ary relation $R \subseteq \N^n$ is partially decidable if and only if there is a $(n+1)$-ary decidable relation $Q \subseteq \N^{n+1}$ such that for any $\vec x \in \N^n$,
    \[
      R(\vec x) \iff \exists y [Q(\vec x,y)].
    \]
\end{result}
\begin{pf}
  ($\Rightarrow$) Suppose $R$ is partially decidable. Then $R = W_e^{(n)}$ for some code $e \in \N$. We then have that
    \[
      R(\vec x) = \varphi_e^{(n)}(\vec x) \da \iff \exists y [T_n(e,\vec x,y)]
    \]
  where $T_n$ is the Kleene T predicate. (To be more explicit, deifne $Q(\vec x,y)$ to hold if and only if $T_n(e, \vec x, y)$.)

  ($\Leftarrow$) Suppose $R$ is a rleation satisfying
    \[
      R(\vec x) \iff \exists y[Q(\vec x,y)]
    \]
  for a decidable $(n+1)$-ary relation $Q \subseteq \N^{n+1}$. Define $f(\vec x) = \mu y[Q(\vec x,y)]$. Then $f$ is computable and $R = \Dom(f)$, so $R$ is partially decidable.
\end{pf}

The next theorem gives closure properties of the class of partially decidable relations.

\begin{result}{theorem}
  The class of partially decidable relations is closed under $\land$, $\lor$, $\exists^<$, $\forall^<$, and $\exists$. It is not closed under negation $\lnot$. More formally:
  \begin{enumerate}[(a)]
    \item If $R$ is partially decidable $n$-ary relations, then the relations $(R \land Q$) and $(R \lor Q)$ are also partially decidable $n$-ary relations.

    \item If $R$ is a partially decidable $(n+1)$-ary relation, then the relations
      \begin{align*}
        (\exists^< R)(\vec x,y) \quad &\iff_\df \quad (\exists z < y) [R(\vec x,z)], \\
        (\forall^< R) (\vec x,y) \quad &\iff_\df \quad (\forall z <y)[R(\vec x,z)], \\
        (\exists R)(\vec x) \quad &\iff_\df \quad \exists y [R(\vec x,y)],
      \end{align*}
    are also partially decidable.

    \item There are partially decidable relations $R$ such that $\lnot R$ is not partially decidable.
  \end{enumerate}
\end{result}
\begin{pf}
  This is annoying. Maybe later lmao.
  % We list this proofs in parts for ease of reading.
  %
  % \begin{enumerate}[(a)]
  %   \item ($\land$) Let $R = \Dom(f)$ or $Q = \Dom(g)$ for some computable partial functions $f,g \colon \N^n \pto \N$. Then, $(R \land Q) = \Dom(f+g)$.
  %
  %   ($\lor$) \textbf{How ot prove the closure under or?}
  %
  %   \item ($\exists^<$)
  %
  %   ($\forall^<$)
  %
  %   ($\exists$) Since $R$ is partially decidable, there exists a decidable relations $Q$ such that $R(\vec x,y)$ holds if and only if $\exists z [Q(\vec x,y,z)]$. Then,
  %     \begin{align*}
  %       (\exists^< R) (\vec x,y) &\iff (\exists z<y)[R(\vec x,z)] \\
  %         &\iff (\exists z < y)(\exists w)[Q(\vec x,z,w)] \\
  %         &\iff (\exists a)[(a)_0 < y \land Q(\vec x, (a)_0, (a)_1)]
  %     \end{align*}
  %   which is partially decidable since projection, comparison, and $Q$ are decidable/computable. T
  % \end{enumerate}
\end{pf}


\begin{result}{corollary}
  If $R$ is an $(n+k)$-ary partially decidable relation, then the $n$-ary relation
    \[
      Q(\vec x) \quad \iff_\df \quad (\exists y_1)\cdots(\exists y_k)[R(\vec x,y_1, \ldots, y_k)]
    \]
  is also partially decidable.
\end{result}
\begin{rem}
  We can rewrite our bunch of existentials as a single number being projected many times, like $Q(\vec x) \iff (\exists a)[R(\vec x, (a)_0, \ldots, (a)_{k-1})]$.
\end{rem}

Here are some more partially decidable relations using these properties.
\begin{ex}
  The following relations are partially decidable for each fixed $n$.
  \begin{enumerate}[(1)]
    \item $R^{(n)}(e) \quad \iff_\df \quad W_e^{(n)} \neq \es$.
      \begin{pf}
        We may rewrite our relation as:
          \[
            W_e^{(n)}\neq \es \iff (\exists x_0) \cdots (\exists x_{n-1})[\varphi_e^{(n)}(\vec x) \da],
          \]
        This is partially decidable since partialy decidability is closed under many existentials and $\varphi_e^{(n)}(\vec x) \da$ is partially decidable. Thus $R^{(n)}$ is partially decidable.
      \end{pf}

    \item $Q^{(n)}(e,y) \quad \iff_\df \quad y \in E_e^{(n)}$.
    \begin{pf}
       Same many existential trick again.
    \end{pf}
  \end{enumerate}
\end{ex}

Now here comes a way to relate decidable and partially decidable relations.
\begin{result}{theorem}
  Let $R$ be an $n$-ary relation. The relation $R$ is decidable if and only if $R$ and $\lnot R$ are both partially decidable.
\end{result}
\begin{pf}
  ($\Rightarrow$) Since $R$ is decidable, $\lnot R$ is also decidable. Then by Example \ref{ex:decidable-implies-partially-decidable}, $R$ and $\lnot R$ are partially decidable.

  ($\Leftarrow$) Suppose $R$ and $\lnot R$ are partially decidable. Let $f,g \colon \N \pto \N$ be such that
    \[
      R(\vec x) \iff f(\vec x)\da \quad \text{and} \quad \lnot R(\vec x) \iff g(\vec x) \da.
    \]
  Notice that for any $\vec x$, $R(\vec x)$ holds (and hence $f(\vec x) \da$) or $\lnot R(\vec x)$ holds (and hence $g(\vec x)\da$), but not both! Then, running both $f(\vec x)$ and $g(\vec x)$ and seeing which halts is a decision procedure for $R$ since if $f(\vec x)$ halts, $R$ holds, and if $g(\vec x)$ halts, $R$ does not hold.
\end{pf}

\begin{result}{corollary}
  Any relation $R$ which is partially decidable but not decidable has $\lnot R$ not partially decidable.
\end{result}

\subsubsection{Graphs of computable partial functions}

Now let's talk about graphs! When we talked about total functions, we saw that a total function was computable if and only if its graph was decidable. It turns out that we actually have an analagous result for partial functions. Indeed, it is:
\begin{result}{theorem}
  Let $f\colon \N^n \to \N$ be a partial function. Then, $f$ is computable if and only if its graph relation $G_f$ is partially decidable.
\end{result}
\begin{pf}
  ($\Rightarrow$) Suppose $f$ is computable and has a code $e \in \N$. Then,
    \[
      f(\vec x) \da  = y \iff \exists z [T_n(e,\vec x,z) \land U(z) = y]
    \]
  which is partially decidable since the statement being quantified over is indeed decidable.

  ($\Leftarrow$) Suppose $G_f$ is partially decidable. Pick a decidable relation $R$ such that
    \[
      G_f(\vec x,y) \iff \exists z [R(\vec x,y,z)].
    \]
  We may then define $f$ by
    \[
      f(\vec x) = (\mu a[R(\vec x, (a)_0, (a)_1)])_0.
    \]
  Thus $f$ is computable since it is defined through minimalization with a decidable relation and substitution with computable functions.
\end{pf}











\subsection{Recursively enumerable sets}

Before we start we're going to do some quick review/comments.
\begin{itemize}
  \item Unary relations on $\N$ are the same things as subsets of $\N$.
  \item Another name for decidable unary relations on $\N$ are called \deft{computable} or \deft{recursive} sets.
  \item A subset $A$ of $\N$ is decidable/computable/recursive if and only if its characteristic function is computable.
  \item A subset $A$ of $\N$ is partially-decidable/semi-computable/semi-recursive if and only if its partial characteristic function is computable.
  \item All of the theorems of the previous section about partially decidable relations apply to these partially decidable sets (unary relations). For example: (1) if $A$ and $B$ are partially decidable subsets of $\N$, then $A \cup B$ and $A \cap B$ are partially decidable (since partial decidability is closed under $\land$ and $\lor$); and (2) a subset $A$ of $\N$ is decidable if and only if $A$ and $\N \sm A$ are partially decidable.
\end{itemize}
\begin{rem}
  That last set of names are kind of garbage imo, so we're going to stick to partially decidable (until we learn about a shorter name).
\end{rem}

Here's an example using an something we know but with this new set up.
\begin{ex}
  The unary, diagonal Halting Problem as a set is usually denoted by $K \ceq \set{x \in \N : \varphi_x(x) \da}$. We know from Example \ref{ex:unary-diagonal-halting-is-partially-decidable} that $K$ is partially decidable.
\end{ex}

Here's the name we were alluding to earlier.
\begin{defn}
  A set $A \subseteq \N$ is \deft{recursively enumerable} (or \deft{computably enumerable}) if and only if either $A$ is empty or there exists a computable total function $f\colon \N \to \N$ such that $A = \Ran(f)$.
\end{defn}
\begin{rem}
  We say $f$ enumerates $A$ since
    \[
      A = \set{f(0), f(1), f(2), \ldots}.
    \]
  Notice that our list above may have repeats.
\end{rem}
\begin{rem}
  We often abbreviate recursively enumerable by r.e. (which we will do from now on).
\end{rem}

Now here's why we said we may be able to use this new name instead of partially decidable.
\begin{result}{theorem}
  A set $A \subseteq \N$ is partially decidable if and only if it is r.e.
\end{result}
\begin{pf}
  ($\Leftarrow$) Suppose $A$ is r.e. Then there is a computable total function $f \colon \N \to \N$ such that $A = \Ran(f)$. Then,
    \[
      y \in A \iff \exists x [f(x)=y].
    \]
  Since $f$ is computable total, its graph is decidable, so $A$ is partially decidable.

  ($\Rightarrow$) Suppose $A$ is partially decidable. If $A$ is empty or finite, it is clearly r.e., so we may assume that $A$ is infinite. Then there exists a decidable relation $R \subseteq \N^2$ such that
    \[
      x \in A \iff \exists y [R(x,y)].
    \]
  Now define $f \colon \N \to \N$ by
    \[
      f(n) = (\mu z [ z \geq n \land R((z)_0,(z)_1)])_0.
    \]
  If $x \in A$, let $z = \cyc{x,y}$ such that $R(x,y)$. Then $f(z) = x$. \textbf{Proof read}
\end{pf}

Here's basically a list of all our equivalent definitions of r.e. that we have established.
\begin{result}{theorem}
  For a set $A \subseteq \N$, the following are equivalent:
  \begin{enumerate}[(i)]
    \item $A$ is r.e.
    \item $A$ is partially decidable (i.e., the partial characteristic function of $A$ is computable).
    \item $A = W_e$ for some $e \in \N$.
    \item There is a decidable binary relation $R$ such that for every $x \in \N$,
      \[
        x \in A \iff \exists y [R(x,y)].
      \]
    \item There is a partially decidable $(n+1)$-ary relation $R$ such that for every $x \in \N$,
      \[
        x \in A \iff \exists y_1 \cdots \exists y_n [R(x,y_1, \ldots, y_n)].
      \]
    \item $A$ is finite or $A$ is the range of a computable injective total function.
    \item $A = E_e$ for some $e \in \N$.
  \end{enumerate}
\end{result}

\begin{ex}
  The set $\set{x \in \N : \varphi_x \text{ is not total}}$ is not r.e.

  \begin{pf}
    We will computably reduce membership in $\clos K \ceq \N \sm K$ to membership in $A \ceq \set{x \in \N : \varphi_x \text{ is not total}}$. (i.e., we want a computable total $f\colon \N \to \N$ such that $x \in \clos K \iff f(x) \in A$.)

    Define
      \[
        g(x,y) =
          \begin{cases}
            0 & \text{if } \varphi_x(x)\da, \\
            \ua & \text{otherwise}.
          \end{cases}
      \]
    Since $\varphi_x(x)\da$ is partially decidable, $g$ is computable. By the s-m-n Theorem, there is a computable total function such that
      \[
        \varphi_{f(x)}(y) = g(x,y)
      \]
    for all $x,y \in \N$. Then we have that
      \[
      x \in \clos K &\iff \varphi_x(x) \ua\\
        \iff \forall y [g(x,y) \ua] \\
        \iff W_{f(x)} = \es \\
        \iff \varphi_{f(x)} \text{ is not total}.
      \]
    Thus if $A$ was r.e., then $\clos K$ would be r.e. which is absurd! Thus $A$ is not r.e. \textbf{ASK: the last equivalence doesn't make sense to me.}
  \end{pf}
\end{ex}
\begin{rem}
  For a subset $A$ of $\N$, we denote its complement by $\clos A$ (i.e., $\clos A \ceq \N \sm A$).
\end{rem}

\begin{result}{theorem}{\label{thm:increasing-enumeration-is-recursive}}
  Let $A \subseteq \N$ be an infinite set. $A$ is recursive if and only if $A$ is the range of a computable total function which is strictly increasing (i.e., $A = \set{f(0) < f(1) < f(2) < \cdots}$).
\end{result}
\begin{pf}
  ($\Rightarrow$) Suppose $A$ is infinite and recursive. Define $f\colon \N \to \N$ by recusion as follows: let $w_0$ be the least element of $A$ and define $f$ by
    \[
      \begin{cases}
        f(0) &= w_0, \\
        f(n+1) &= \mu x [x \in A \land x > f(n)].
      \end{cases}
    \]
  Notice that $f$ is then computable since it is defined from recursion by computable functions (in particular, a minimalization of a decidable relation). We know that $f$ enumerates $A$ since find a new element of $A$ with every new value of $f$.

  ($\Leftarrow$) Suppose $f$ is computable and enumerates $A$ in increasing order. Then $f$ is strictly increasing, so $f(n) \geq n$. Then apply our midterm problem.
\end{pf}

\begin{result}{theorem}
  Every infinite r.e. set has an infinite recursive subset.
\end{result}
\begin{pf}
  Let $A$ be an infinite r.e. set. Pick a computable total function $f\colon \N \to \N$ such that $A = \Ran(f)$.

  Define $g\colon \N \to \N$ by
    \[
      \begin{cases}
        g(0) &= f(0) \\
        g(n+1) &= f(\mu x [f(x) > g(n)]).
      \end{cases}
    \]
  Our function $g$ is computable since we are composing computable functions (the inner argument is computable since it is the minimalization of a decidable comparison). Furthermore, $g$ is total since $A$ is infinite. Finally, because $g$ is strictly increasing, $\Ran(g) \subseteq A$ is recursive by Theorem \ref{thm:increasing-enumeration-is-recursive}. 
\end{pf}

\end{document}
