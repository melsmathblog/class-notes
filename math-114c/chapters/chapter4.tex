\documentclass[class=article, crop=false]{standalone}
\input{preamble}

\begin{document}

\section{Relativized computing}

\subsection{Oracle machines}

Let $\alpha \colon \N\to\N$ be a total function. $\alpha$ is an arbitrary functions, perhaps not computable.

We consider an oracle machine with oracle $\alpha$:

\begin{figure}[ht]
  \center
  \resizebox{\textwidth}{!}{\figpath{oracle-diagram.pdf_tex}}
  \label{fig:oracle-diagram}
\end{figure}

This motivates the idea of computable relative to $\alpha$.

\subsubsection{Unlimited register machine with oracle}

An unlimited register machine with oracle (URMO) is the same as previous register machines except there is \underline{one more type of instruction:}

The machine responds to instruction $O(n)$ by replacing $r_n$, the contents of the register $R_n$, with the oracle function's value at $r_n$, $\alpha(r_n)$.

\textbf{insert example maybe? finish}

A URMO program is a finite ordered list of instructions of the form $Z(n), S(n), T(m,n), J(m,n,q)$, and $O(n)$.

\begin{rem}
  (1) If in a URMO program $P$, the instruction $I_k$ is an oracle instruction $O(n)$, then after executing $I_k$ the next instruction to be executed is $I_{k+1}$.

  (2) The oracle function $\alpha$ is not a part of the machine, but rather a "plug-in" or "external" device that can be swapped for another oracle $\beta\colon \N\to\N$.

  (3) Every URM program is an URMO program (since URM programs are just URMO programs that never use the oracle instruction.)
\end{rem}

\begin{defn}
  For a positive integer $n$ and an oracle $\alpha$ and a URMO program $P$, we define the partial function $f_P^{\alpha,n} \colon \N^n \pto \N$ as follows:  for input $\vec x = (x_1,\ldots,x_n) \in \N^n$, run program $P$ on the URMO with oracle $\alpha$ with the initial state given by $\vec x$ and if the machine halts, then $f_P^{\alpha,n}(\vec x) \da = $ the contents of register $R_1$ in the halting state.
\end{defn}
\begin{rem}
  This is just us redefining programs as partial functions in our new type of computing.
\end{rem}

\begin{defn}
  An $n$-ary partial function $g\colon \N^n \pto \N$ is called $\alpha$-computable if and only if $g=f_P^{\alpha,n}$ for some URMO program $P$.
\end{defn}

\begin{ex}
  \begin{enumerate}[(1)]
    \item Every computable partial function is $\alpha$-computable for every oracle $\alpha$.

      \begin{pf}
        URM prgorams are URMO programs.
      \end{pf}


  ` \item `$\alpha$ is $\alpha$-computable.

      \begin{pf}
        A URMO program that computes $\alpha$ is $O(1)$.
      \end{pf}
  \end{enumerate}
\end{ex}


\subsubsection{Another approach: $\alpha$-recursive partial functions}

\begin{defn}
  The class of $\alpha$-recursive partial functions is the smallest class of partial functions which
    \begin{enumerate}[(a)]
      \item contains the basic functions: zero function, successor function, projection functions;
      \item contains $\alpha$;
      \item is closed under definition by substitution, recursion, and minimalization.
    \end{enumerate}
\end{defn}


\begin{result}{theorem}
  A partial function $f\colon \N^n \to \N$ is $\alpha$-computable if and only if $f$ is $\alpha$-recursive.
\end{result}
\begin{pf}
  We won't prove this but the proof in outline is exactly the same as the non-relativized version.
\end{pf}

\begin{ex}
  If $\alpha,\beta \colon \N \to \N$ are both oracles and $\alpha$ is $\beta$-recursive, then every $\alpha$-recursive partial function is also $\beta$-recursive.
    \begin{pf}
      Suppose $\alpha$ is $\beta$-recursive. Then $\mathcal C^\beta \ceq $ the class of $\beta$-recursive partial functions contains the basic functions, $\alpha$, and is closed under substitution, recursion, and minimalization, but $\mathcal C^\alpha$ is the smallest class with these properties, so $\mathcal C^\alpha \subseteq \mathcal C^\beta$.
    \end{pf}

  In particular: if $\alpha$ is computabel, then a partial function is $\alpha$-recursive if and only if it is recursive.


\end{ex}











\end{document}
