\documentclass[class=article, crop=false]{standalone}
\input{preamble}

\begin{document}

\section{Lecture 6}

\textit{Recap:} $f \colon \R^n \to \R^n$ is (locally) Lipschitz if for all $x,y \in D \subseteq \R^n$,
    \[
        \frac{|f(x)-f(y)|}{|x-y|} \leq L = \text{constant}.
    \]
If $D = \R^n$, then $f$ is globally Lipschitz.

Consider
    \[
        y'(t)=f(y); y(t_0) = y_0 \tag{1}.
    \]

\begin{enumerate}[$\bullet$]
    \item $J = [t_0-a,t_0+a]$
    \item Define $T$ on $C(J)$ as
        \[
            Tu(t) = y_0 + \int_{t_0}^{t} f(u(\tau)) \mathrm{d}\tau; Tu \in C(J)
        \]

    \item Let $\clos B(y_0;b) \coloneqq \set{x : |y_0-x| \leq b} = $ closed ball (in $\R^n$) around $y_0$ of radius $b$.

    \item $| \cdot |$ - Euclidean norm $= \sqrt{x_1^2+ \cdots + x_n^2}$.
\end{enumerate}

\begin{thm}[Picard's Theorem]
    Suppose $y_0 \in \R^n$, $f$ is locally Lipschitz on $\clos B(y_0;b)$ with constant $k$. Then the IVP (1) has a solution $y(t)$ for $t \in J$ provided $a \leq \frac{b}{M}$, where
        \[
            M \coloneqq \max \set{|f(x)| : x \in \clos B(y_0;b)}.
        \]
\end{thm}
\begin{rem}
    We say $\max$ and not $\sup$ because $\clos B(y_0;b)$ is compact, so since $f$ is continuous it achieves its max and min.
\end{rem}

\begin{thm}[Non-autonomous Case:]
    Let (1) $f(t,y)$ be continuous in $t$ and (2) $f(t,y)$ be locally Lipschitz in $y$ for $y \in \clos B(y_0;b)$ and $t \in [t_0-c, t_0+c]$.

    Then, there exists some $a > 0$ such that the IVP (1) has a unique solution over $[t_0-a,t_0+a]$.

    In particular, $a = \min \set{\frac{b}{M},c}$, $M = \max \set{|f(y)| : y \in \clos B(y_0;b)}$.
\end{thm}

\begin{rem}
    (1) $f$ is locally Lipschitz in the dependent ???
\end{rem}

\begin{ex}
    $f(t,y) = ty$, $y \in D$, $t \in (-c,c)$. $f$ is Lipschitz Continuous (LC) in $y$.
        \[
            \frac{|f(x)-f(y)|}{|x-y|} = \frac{|tx-ty|}{|x-y|} = |t| \leq c
        \]
\end{ex}

\begin{rem}
    Picard's Theorem guarantees \textit{local existence and uniqueness}. This is \textbf{not a necessary condition}.
\end{rem}

\begin{rem}
    A word of caution: regarding the statement of Picard's Theorem as stated in the textbook (Section 70, Theorem A). They say derivative continuous and then ease then ease the condition to LC. (It works the same as our theorem stated in lecture but their's is lame.)
\end{rem}











\newpage
\section{Computability Lecture 6}

\begin{thm}[Substitution]
    Suppose $f \colon \N^k \pto \N$ and $g_1, \ldots, g_k \colon \N^n \pto \N$ are computable partial functions. Then the composition $h \colon \N^n \pto \N defined by $
\end{thm}


\end{document}
