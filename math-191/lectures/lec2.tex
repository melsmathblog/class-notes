\documentclass[class=article, crop=false]{standalone}
\input{prepath}
\input{\prepath}

\begin{document}

\section{Lecture 2---March 31, 2021}

\subsection*{Recall}
Last time introduced categories. Categories are composed of objects and morphisms. Morphisms are composed of domain and codomain, identity maps, associativity, and unital (i.e., identity acts like identity).

\begin{understandingcheck}{example}
  $\Sf{Set}$, $\Sf{Set}_*$, $\Sf{Mon}$, $\Sf{Grp}$, $\Sf{Preord}$, etc. Also, subcategories: $\Sf{CMon} \subseteq \Sf{Mon}$, $\Sf{Ab} \subseteq \Sf{Grp}$, $\Sf{Pos} \subseteq \Sf{Preord}$.
\end{understandingcheck}

Notice that the categories in this example are collections of ``structured sets" plus structure preserving maps between them.

\begin{understandingcheck}{question}
  Can we encode familiar properties of functions categorically?
\end{understandingcheck}

\begin{defn}
  Suppose that $\Cc C$ is a category. A morphism $f\colon A\to B$ in $\Cc C$ is an \emph{isomorphism} if it has a two-sided inverse, i.e., if there exists a morphisms $g\colon B \to A$ such that $g\circ f = 1_A$ and $f \circ g = 1_B$.

  Two objects $A,B \in \Cc C$ are \emph{isomorphic} if there is an isomorphism between them, in which case, we write $A \cong B$.
\end{defn}

\begin{understandingcheck}{example}
  In each of the categories $\Sf{Set}$, $\Sf{Set}_*$, $\Sf{Mon}$, $\Sf{Grp}$, $\Sf{Preord}$, an isomorphism is just a bijective set map that preserves all structure (operations \& distinguished elements).
\end{understandingcheck}

\begin{understandingcheck}{example}
  In each of categories $\Sf{Preord}$ and $\Sf{Pos}$, an isomorphism $f\colon P \to Q$ is a bijective set map such that
    \[
      x\leq y \iff f(x) \leq f(y).
    \]
\end{understandingcheck}

\textbf{Upshot:} Purely categorical condition of invertibility (having a two-sided inverse) encodes bijective correspondences that identify structure (in these ``concrete" cases).

\subsection*{Encoding injectivity and surjectivity}

\begin{defn}
  Let $\Cc C$ be a category. A morphism $f\colon A \to B$ in $\Cc C$ is a \emph{monomorphism} if: for any $T \in \Cc C$ and $h,k\colon T \rightrightarrows A$ if $fh = fk$, then $h=k$.

  Similarly, a morphism $f\colon A \to B$ in $\Cc C$ is an \emph{epimorphism} if for any $T \in \Cc C$ and $h,k \colon B \to T$, if $hf = kf$, tehn $h=k$.
\end{defn}

\begin{understandingcheck}{example}
  Suppose $f\colon A\to B$ is in $\Sf{Set}$.Then $f$ is monomorphism if and only if $f$ is injective, and $f$ is epimorphism if and only if $f$ is surjective.
\end{understandingcheck}
\begin{rem}
  These don't correspond in all categories.
\end{rem}

We can say a bit more in $\Sf{Set}$. If $r\colon A \to B$ is and epimorphism in $\Sf{Set}$, then there is a function $s\colon B \to A$ such that $r \circ s = 1_B$.

\begin{defn}
  Suppose $r\colon A \to B$ and $s\colon B \to A$ are morphisms such that $r \circ s = 1_B$. Then $s$ is a \emph{section} or \emph{right inverse} to $r$, and $r$ is a \emph{retraction} or \emph{left inverse} to $s$. In general, we call a morphism a \emph{split epimorphism} if it has a section, and a \emph{split monomorphism} if it has a retraction.
\end{defn}
\begin{rem}
  Split monomorphism $\implies$ monomorphism.
\end{rem}
\begin{rem}
  Split epimorphism $\implies$ epimorphism.
\end{rem}
\begin{rem}
  Isomorphism $\implies$ split monomorphism and split epimorphism $\implies$ monomorphism and epimorphism

  \textbf{WARNING:} converse is false in general.
\end{rem}

\subsection*{Categories as algebraic objects}

We can view categories as algebraic objects.

We can regard a number of familiar mathematical objects as categories.
\begin{understandingcheck}{example}
  A \emph{preorder category} is a category $\Cc P$ with at most one morphism in $\Cc P(A,B) \ceq \set{f\colon A \to B}$ for any $A,B \in \Cc P$. If $\Cc P$ is a preorder category with only a $\Sf{Set}$'s worth of morphisms, and we define
    \[
      A \leq B \iff \exists f\colon A \to B
    \]
  for all $A,B \in \Cc P$, then $(\operatorname{Ob}(\Cc P), \leq)$ is a preorder. Conversely, if $(P, \leq)$ is a preorder, and we define a category $\Cc P$ by taking  $\operatorname{Ob}(\Cc P) = P$ and $\operatorname{Mor}(\Cc P) \ceq \set{(A,B) \in P \x P : A \leq B}$ and set $\Dom(A,B) = A$ and $\Cod(A,B) = B$, $\operatorname{id}_A = (A,A)$, and $(B,C) \circ (A,B) = (A,C)$, then we get a preorder category.

  Preorders and preorder categories are basically the same thing.
\end{understandingcheck}

\begin{understandingcheck}{example}
  A \emph{monoid category} (nonstandard terminology) is a category with a single object:
    \[
      \begin{tikzcd}
        A \arrow[loop, distance=2em, in=125, out=55] \arrow[loop, distance=2em, in=35, out=325] \arrow[loop, distance=2em, in=215, out=145]
      \end{tikzcd}
    \]

  If $\Cc M$ is a monoid category with object $A \in \Cc M$, then $(\Cc M(A,A), \circ, 1_A)$ is a monoid. Conversely, if $(M, \cdot, e)$ is a monoid, then the category $\Sf{BM}$, with single object $*$ and morphisms $\Cc M$ (all with domain/codomain $*$) is a category with $1_* = e$ and $y\circ x = y \cdot x$.
    \[
      \begin{tikzcd}
        * \arrow["\boxed M"', loop, distance=2em, in=125, out=55]
      \end{tikzcd}
    \]

\end{understandingcheck}

\begin{understandingcheck}{example}
  A \emph{group category} (nonstandard terminology) is a monoid category $\Cc G$ such that every morphism of $\Cc G$ is an isomorphism. The same constructions relating monoid and monoid categories restrict to constructions relating groups and group categories.
\end{understandingcheck}

These examples justify the following terminology:
\begin{defn}
  A \emph{groupoid} is a category in which every morphism is an isomorphism.
    \[
    \begin{tikzcd}
      \bullet \arrow[r, "\cong"] \arrow[d, "\cong", bend left]   & \bullet \\
      \bullet \arrow[u, "\cong", bend left] \arrow[ru, "\cong"'] &
    \end{tikzcd}
    \]
\end{defn}
\begin{rem}
  Groupoid is $\approx$ ``many object group".
\end{rem}

\begin{defn}
  The \emph{core} of a category $\Cc C$ is the subcategory with objects consisting of all objects of $\Cc C$ and morphisms consisting of all isomorphisms in $\Cc C$.
\end{defn}

\subsection*{Size consideration}

\emph{Russel's Paradox:} there is not set of all sets.

The collection of all sets is ``too large" to be a set. (Grothendieck universes can be used to stratify sets by their size, so we have ``small sets", ``large sets", ``even larger sets", etc.)

\begin{defn}
  Let $\Cc C$ be a category. We say that $\Cc C$ is \emph{small} if $\operatorname{Mor}(\Cc C)$ is a set (also implies that the $\operatorname{Ob}(\Cc C)$ is a set). We say that $\Cc C$ is \emph{locally small} if $\Cc C(A,B) \ceq \set{f\colon A \to B}$ is a set for all $A,B\in \Cc C$. A category $\Cc C$ is \emph{large} if it is \textbf{not} small.
\end{defn}

\begin{understandingcheck}{example}
  $\Sf{Set}$, $\Sf{Set}_*$, $\Sf{Mon}$, $\Sf{Grp}$, $\Sf{Preord}$, etc. are locally small, large categories.
\end{understandingcheck}

\begin{understandingcheck}{example}
  The preorder category associated with a preorder, monoid category associated to a monoid, and a group category associated to a group are small categories.
\end{understandingcheck}


\end{document}
