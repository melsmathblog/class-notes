\documentclass[class=article, crop=false]{standalone}
\input{preamble}

\externaldocument{chapter1}

\begin{document}

\section{Basics of topology}

\subsection{Super basics of topological spaces}

We start with the titular definition.
\begin{defn}[Topology]
  A \deft{topology} on a set $X$ is a collection $\mathcal T$ of subsets of $X$ such that
    \begin{enumerate}[(1)]
      \item $\es, X \in \mathcal T$;
      \item any union of sets in $\mathcal T$ is in $\mathcal T$; and
      \item any finite intersection of sets in $\mathcal T$ is in $\mathcal T$.
    \end{enumerate}
  The elements of $\mathcal T$ are called \deft{open sets}. A topological space is a pair $(X,\mathcal T)$.
\end{defn}
\begin{rem}
  A topology is a prescription for which sets are open.
\end{rem}

Let's relate this back to something more familiar and intuitive: open sets in metric spaces. Recall the following properties of open sets in metric spaces: the empty set and the whole space are open; any union of open sets is open; and any finite intersection of open sets is open. I think of our definition of topology as a generalization of that.

Now let's get to some examples of topologies. (Some are gross, which seems to be a common theme in topology. Things can get pretty pathological at times.)

\begin{ex} \ldr
  \begin{enumerate}[(1)]
    \item If $X$ is a metric space, the collection of open sets is a topology called the \deft{metric topology}.

    \item If $X$ is any set, the family $\mathcal T \coloneqq \set{\es, X}$ forms a topology called the \deft{indiscrete topology}.

    \item If $X$ is any set, the family $\mathcal T \coloneqq \mathcal P (X)$ (the collection of all subsets) forms the \deft{discrete topology}.

    \item Let $X \coloneqq \set{a,b}$. Then $\mathcal T \coloneqq \set{\es, \set a, X}$ is a topology.
  \end{enumerate}
\end{ex}

Let's start translating some of our basic metric-space-type ideas to more general topological spaces. Next we have the neighborhood (another name for an open ball in metric spaces).

\begin{defn}[Neighborhood]
  Let $X$ be a topological space. An open \deft{neighborhood} of $x \in X$ is an open set $U$ with $x \in U$.
\end{defn}

Recall that we would define open sets in metric spaces using interior points. We actually have a very similar idea in general topological spaces using open neighborhoods instead of the open ball.

\begin{defn}
  Let $X$ be a topological space and suppose $S$ is a subset of $X$. A point $x \in X$ is an \deft{interior point} of $S$ if and only if there exists an open neighborhood $U$ of $x$ that is contained in $S$

  The \deft{interior} of $S$ is the set of interior points of $S$ and is denoted $S^\circ$.
\end{defn}

\begin{result}{proposition}{\label{prop:interior-is-open-and-open-iff-set-is-interior}}
  Let $X$ be a topological space and let $S$ be a subset of $X$. Then
  \begin{enumerate}[(1)]
    \item $S^\circ$ is open and
    \item $S$ is open if and only if $S = S^\circ$.
  \end{enumerate}
\end{result}
\begin{pf}
  (1) Suppose $s \in S^\circ$. Then there exists an open neighborhood $U_s$ of $s$ contained in $S$. In fact, $U_s$ is an open neighborhood of all elements of $S$ in $U_s$. Hence, all points of $U_s$ are interior points of $S$ and we have that $U_s \subseteq S^\circ$. Now, notice that $\bigcup_{s \in S^\circ}^{} U_s$ is open by properties of open sets. Moreover, notice that
    \[
      S^\circ \subseteq \bigcup_{s \in S^\circ}^{} U_s \subseteq S^\circ \implies S^\circ = \bigcup_{s \in S^\circ}^{} U_s,
    \]
  so $S^\circ$ is open.

  (2) ($\Rightarrow$) Suppose $S$ is open. Then $S$ is an open neighborhood of any of its points, so $S \subseteq S^\circ \subseteq S$. Thus $S^\circ = S$. ($\Leftarrow$) If $S = S^\circ$, then $S$ is open since $S^\circ$ is open.
\end{pf}

Now let's get to closed sets. (They're what you think they are, if you're familiar with metric topology.)

\begin{defn}[Closed set]
  Let $X$ be a topological space. A subset $C$ of $X$ is \deft{closed} if and only if $X\sm C$ is open.
\end{defn}
\noindent It turns out that our adherent points thing from metric spaces still works but this is the more topology-core definition of closed. (We'll see this more and more in topology as we continue to study it, but things are usually defined in terms of open sets.) Let's actually get into adherent points now.

\begin{defn}[Adherent point]
  Let $X$ be a topological space. A point $x \in X$ is \deft{adherent} to a subset $S$ of $X$ if and only if every open neighorhood of $x$ has nonempty intersection with $S$.
\end{defn}

Now that we have that, we can get to closures.

\begin{defn}[Closure of a set]
  Let $X$ be a topological space and let $S$ be a subset of $X$. The \deft{closure} of $S$ is defined to be $X \sm (X \sm S)^\circ$ and is denoted $\clos S$. Moreover, $\clos S$ consists precisely of the points of $X$ adherent to $S$.
\end{defn}

\begin{result}{proposition}{\label{prop:closure-is-closed}}
  The closure of a set is closed.
\end{result}
\begin{pf}
  This is clear considering that the closure is in fact the complement of an interior (which is always open).
\end{pf}

\begin{result}{proposition}
  Let $X$ be a topological space and let $S$ be a subset of $X$. Then, $S$ is closed if and only if $S = \clos S$.
\end{result}
\begin{pf}
  ($\Leftarrow$) Suppose $S  = \clos S$. Then $S$ is closed (by Proposition \ref{prop:closure-is-closed}).

  ($\Rightarrow$) Suppose $S$ is closed. Then $X \sm S$ is open (by definition), so $(X \sm S)^\circ = X \sm S$ by Proposition \ref{prop:interior-is-open-and-open-iff-set-is-interior}. Thus we have that
    \[
      \clos S = X \sm (X \sm S)^\circ = X (X \sm S) = S.
    \]
\end{pf}

Now let's move on to sequences.

\begin{defn}
  A sequence of points $(x_n)$ in a topological space $X$ \deft{converges} to $x \in X$ if and only if for every open neighborhood $U$ of $x$, there exists an integer $N$ such that $x_n \in U$ whenever $n \geq N$.
\end{defn}

Here's a neat little theorem.

\begin{result}{theorem}
  If $S$ is a subset of a topological space $X$ and $(x_n)$ is a sequence in $S$ converging to $x \in X$, then $x \in \clos S$.
\end{result}
\begin{pf}
  Suppose towards a contradiction that $x \nin \clos S$. Then $x \in X \sm \clos S$, which is open. Then there exists an open neighborhood of $x$ entirely contained in $X \sm \clos S$ and since $(x_n)$ converges to $x$, such a neighborhood contains infinitely many terms of $(x_n)$. But $x_n \in S$ for all $n$.
\end{pf}

Now let's talk a bit more about adherent points. It turns out that we can use adherent points to define what the boundary of a subset of a topological space is.
\begin{defn}
  Let $X$ be a topological space and let $S$ be a subset of $X$. A point $x \in X$ is a \deft{boundary point} of $S$ if and only if $x$ is adherent to both $S$ and $X \sm S$ (i.e., $x \in \clos S \cap \clos{(X \sm S)}$).

  The set of boundary points, or \deft{boundary} of $S$, is $\clos S \cap \clos{(X \sm S)}$ and is denoted by $\partial S$.
\end{defn}

\begin{result}{proposition}
  Let $X$ be a topological space and let $S$ be a subset of $X$. Then,
  \begin{enumerate}[(1)]
    \item $S^\circ \cap \partial S= \es$ and
    \item $\clos S = S^\circ \cup \partial S$
  \end{enumerate}
\end{result}
\begin{pf}
  (1) Suppose towards a contradiction that $S^\circ \cap \partial S$ is nonempty. Then there exists an $x$ in the intersection. Then there exists an open neighborhood of $x$ that is completely contained inside $S$. However, since $x \in \partial S$, every open neighorbood intersects $X \sm S$ nontrivially. Thus we have reached a contradiction and $S^\circ \cap \partial S$ is empty.

  (2) ($\supseteq$) Since $S^\circ \subseteq S \subseteq \clos S$ and $\partial S \subseteq \clos S$, we have that $S^\circ \cup \partial S \subseteq S$. ($\subseteq$) Suppose $x \in \clos S$. Then, either there exists an open neighborhood $U$ of $x$ with $U \subseteq S$ or not. In the former case, $x \in S^\circ$. In the latter case, any open neighorhood of $x$ intersects $X \sm S$, so $x \in \clos{(X \sm S)}$, so $x \in \clos S \cap \clos{(X \sm S)}$, so $x \in \partial S$.
\end{pf}

Now let's move on to I think our shortest section: subspaces.











\subsection{Subspaces}

\begin{defn}
  Let $S$ be a subset of a topological space $(X, \mathcal T)$. The family
    \[
      \mathcal S \coloneqq \set{U \cap S : U \in \mathcal T}
    \]
  is a topology for $S$ called the \deft{subspace topology}. (This is also called the \deft{relative topology}.) The elements of $\mathcal S$ are called \deft{relatively open} subsets of $S$.
\end{defn}

Here's an example of this:
\begin{ex}
  If $X$ is a metric space and $Y$ is a subspace of $X$, the topology on $Y$ obtained by restricting the metric to $Y$ coincides with the relative topology.
\end{ex}










\subsection{Continuous functions}

Because general topological spaces do not have metrics, we can't actually define metrics using our usual $\e$-$\delta$ definition, so we use another idea (that's more topology-core tbh): open sets. In particular, we define a function to be continuous if and only if preimages preserve openness. Let's put this in one of our pretty boxes.
\begin{defn}[Continuity]
  Let $X$ and $Y$ be topological spaces and let $f \colon X \to Y$ be a function.

  The function $f$ is \deft{continuous} if and only if $f^{-1}(V)$ is open whenever $V$ is an open subset of $Y$ (where $f^{-1}(V)$ is the preimage of $V$).

  The function $f$ is \deft{continuous at a point} $x \in X$ if and only if for every open set $V \subseteq Y$ such that $f(x) \in V$, there exists an open set $U \subseteq X$ such that $x \in U$ and $f(U) \subseteq V$.
\end{defn}

\begin{slogan}
  Continuous $\iff$ preimages preserve openness.
\end{slogan}

It turns out that being continuous at every point in your domain does indeed mean you're continuous. Here's it stated formally and its proof.

\begin{result}{theorem}
  Let $X$ and $Y$ be topological spaces. A map $f \colon X \to Y$ is continuous if and only if $f$ is continuous at every point of $X$.
\end{result}
\begin{pf}
  ($\Rightarrow$) Suppose $f$ is continuous. Taken $x \in X$ and let $V$ be an open subset of $Y$ containing $f(x)$. Then, $U = f^{-1}(V)$ is open in $X$ (by definition of being continuous), contains $x$, and satisfies $f(U) \subseteq V$.

  ($\Leftarrow$) Suppose $f$ is continuous at every point of $X$. Let $V$ be an open subset of $Y$ and take $x \in f^{-1}(V)$. Then there exists an open set $U_x \subseteq X$ containing $x$ with $U_x \subseteq f^{-1}(V)$, which means that $f(U_x) \subseteq V$. \textbf{FINISH}
\end{pf}

Honestly this last theorem kind of sucks tbh. I don't see this fact being very useful in my (brief) experience of topology. But now we have something definitely more useful: the composition of continuous functions is continuous.

\begin{result}{theorem}[Composition of continuous functions is continuous]
  Let $X$, $Y$, and $Z$ be topological spaces, $f \colon X \to Y$, and $g \colon Y \to Z$ be cotninuous maps. Then $g \circ f \colon X \to Z$ is continuous.
\end{result}
\begin{pf}
  Let $V$ be an open subset of $Z$. Then $g^{-1}(V) \subseteq Y$ is open and $f^{-1}(g^{-1}(V)) \subseteq X$ is open. Since $(g \circ f)^{-1} = f^{-1} \circ g^{-1}$, we have shown that $g \circ f$ is continuous.
\end{pf}

Here's a definition very related to continuity: openness (of a function).

\begin{defn}[Open function]
  Let $X$ and $Y$ be topological spaces and let $f \colon X \to Y$ be a function. The function $f$ is \deft{open} if and only if for all open subsets $U$ of $X$, the image $f(U)$ is an open subset of $Y$.
\end{defn}

\begin{slogan}
  Open $\iff$ images preserve openness.
\end{slogan}

Now we have a very special type of continuous function: homeomorphisms.
\begin{defn}[Homeomorphism]
  Let $X$ and $Y$ be topological spaces. A function $f \colon X \to Y$ is a \deft{homeomorphism} if and only if $f$ is continuous, bijective, and open.
\end{defn}
\begin{rem}
  For a bijection $f \colon X \to Y$ and topological spaces $X$ and $Y$, $f$ is open if and only if $f^{-1}$ is continuous.
\end{rem}

This coming idea is what we call it when we have a homeomorphism, but it's not immediately relevant I'd say. This will come up more often as we continue to learn about things called \deft{topological properties}, but that's a story for another day (or rather very soon lmao).

\begin{defn}[Homeomorphic spaces]
  Two topological spaces $X$ and $Y$ are \deft{homeomorphic} if and only if there exists a homeomorphism $f \colon X \to Y$.
\end{defn}

% \begin{slogan}
%   Things are homeomorphic $\iff$ they have the same open sets.
% \end{slogan}

\begin{rem}
  If $f \colon X \to Y$ is a homeomorphism, then so is $f^{-1} \colon Y \to X$. Compositions of homeomorphisms are also homeomorphisms and the identity map is a homeomorphism. Thus, ``being homeomorphic" is an equivalence relation on topological spaces.
\end{rem}

\begin{ex}
  In $\R$, let $X = [0,r)$ and $Y=[0,\infty)$. Then
    \[
      f(t) \coloneqq \frac{t}{r-t}
    \]
  is a homeomorphism from $X$ to $Y$ (exercise lmao).

  This can be used to show that in $\R^n$, if $X = B(\vec 0;r)$ and $Y=\R^n$, then $X$ and $Y$ are homeomorphic via $g\colon B(\vec 0;r) \to \R^n$
    \[
      g(\vec x) \coloneqq f(\|\vec x\|) \vec x.
    \]
  More generally, any ball $B(\vec x_0 ;r)$ around any $\vec x_0 \in \R^n$ is homeomorphic to $\R^n$ itself via
    \[
      h(\vec x) \coloneqq f(\|\vec x - \vec x_0\|) (\vec x - \vec x_0).
    \]
\end{ex}

\begin{rem}
  A property of a topological space is a \deft{topological invariant} or \deft{topological property} if and only if it is preserved under homeomorphisms. We'll get some examples of these soon.
\end{rem}









\subsection{Basis for a topology}

We revisit several notions from metric space topology and discuss them in a more general setting.

\textbf{[insert review of basis and stuff from metric space stuff]}

\begin{result}{theorem}
  A family $\BB$ of subsets of $X$ is a basis for a topology of $X$ if and only if $\BB$ satisfies
    \begin{enumerate}[(1)]
      \item Each $x \in X$ lies in at least one set from $\BB$.
      \item If $U,V \in \BB$ and $x \in U \cap V$, then there exists $W \in \BB$ with $x \in W \subseteq U \cap V$.
    \end{enumerate}
\end{result}
\begin{pf}
  ($\Rightarrow$) Assume $\BB$ is a basis. Then $\BB$ satisfies (1) by the recalled theorem above. Moreover, since $U \cap V$ is open, that same theorem shows that $\BB$ satisfies (2).

  ($\Leftarrow$) Assume $\BB$ satisfies (1) and (2). Let $\mathcal T$ be the family of all unions of sets from $\BB$. It suffices to show that $\mathcal T$ is a topology for $X$ (in which case, $\BB$ is basis for $\mathcal T$). To this end, recall that
  \begin{enumerate}[(i)]
    \item $\es, X \in \mathcal T$,
    \item any union of sets in $\mathcal T$ is in $\mathcal T$, and
    \item any finite intersection of sets in $\mathcal T$ is in $\mathcal T$.
  \end{enumerate}
  \noindent (ii) is clear by construction of $\mathcal T$.

  In regards to (i), $\es = \bigcup_{\es}^{} \in \mathcal T$ and the fact that $\BB$ satisfies (1) means that $\bigcup_{V \in \BB}^{} V$, hence $X \in \mathcal T$.

  For (iii), it suffices to show that the intersection of any two sets in $\mathcal T$ is in $\mathcal T$. So let $U,V \in \mathcal T$. Since $U$ and $V$ are unions of sets in $\BB$, there exists $U_0$ and $V_0$ with $x \in U_0 \subseteq U$ and $x \in V_0 \subseteq V$, thus $x \in U_0 \cap V_0$. Since $\BB$ satisfies (2), we know there exists $W_x \in \BB$
  such that $x \in W_x$ and $W_x \subseteq U_0 \cap V_0$. Then $W_x \subseteq U \cap V$ and
    \[
      U \cap V = \bigcup_{x \in U \cap V}^{} W_x
    \]
  hence, $U \cap V \in \mathcal T$.
\end{pf}

It obvious that every topological space has a basis (take the entire topology, for example), but there are some interesting results that we can get from putting constraints on the size of basis.

\begin{defn}[Second countable]
  A topologicals space $X$ is \deft{second countable} if and only if there is a countable basis for the topology of $X$.
\end{defn}

\begin{result}{theorem}[Lindel\"{o}f's Theorem]
  Let $X$ be a second countable topological space. Then every open cover of $X$ has a countable subcover.
\end{result}

\begin{pf}
  Identical as done in the section on compactness of metric spaces.
\end{pf}

\begin{defn}[Dense]
  A subset $S$ of a topological space $X$ is \deft{dense} in $X$ if $\clos S = X$.
\end{defn}

\begin{defn}[Separable]
  A topological space $X$ is separable if and only if there is a countable dense subset of $X$.
\end{defn}

\begin{result}{theorem}
  If $X$ is a second countable topological space, then $X$ is separable.
\end{result}

\begin{rem}
  There are separable topological spaces which are not second countable. Such spaces cannot be metrizable.
\end{rem}










\subsection{Finite product spaces}

What happens when you take products of topological spaces? When we take products of sets it's pretty straight forward, but for topological spaces we have to ensure that that structure of a topology is itself preserved, but how do we manage that? We do that with a thing called the \deft{product topology} which we'll define now.

\begin{defn}[Product topology]
  Let $X_1, \ldots, X_n$ be topological spaces. The \deft{product topology} on the Cartesian product $X \coloneqq X_1 \x \cdots \x X_n$ is teh topology for which a basis given by the ``rectangles:"
    \[
      \set{U_1 \x \cdots \x U_n : U_i \subseteq X_i \text{ is open for } 1 \leq i \leq n}.
    \]
\end{defn}

The product topology ensures that component projections are continuous: indeed, if $\pi_j : X \to X_j$ is the map such that
  \[
    (x_1, \ldots, x_n) \mapsto x_j
  \]
for all $(x_1, \ldots, x_n) \in X$ and $U_j \subseteq X_j$ is open, then
  \[
    \pi_j^{-1}(U_j) = X_1 \x \cdots \x X_{j-1} \x U_j \x X_{j+1} \cdots \x X_n
  \]
which is open in $X$. In fact, the product topology is the smallest such topology that guarantees that the projection functions are continuous. We'll show this now.

\begin{result}{theorem}
  The product topology for $X \coloneqq X_1 \x \cdots \x X_n$ is the smallest topology for which the component projections $\pi_j$ are continuous.
\end{result}
\begin{pf}
  Let $\mathcal T$ be a topology for which the projections $\pi_j$ are $\mathcal T$-continuous. Let $U_1, \ldots, U_n$ be open subset of $X_1, \ldots, X_n$, respectively. We've seen taht each $\pi_j^{-1}(U_j)$ is an open set in $X$. Since
    \[
      \pi_1^{-1}(U_1) \cap \cdots \cap \pi_n^{-1}(U_n) = U_1 \x \cdots \x U_n \subseteq \mathcal T
    \]
  is an open set, $\mathcal T$ contains the basis for the product topology.
\end{pf}
\begin{rem}
  Why does this actually show that the product topology is the smallest? When any topology then contains the basis of the product topology, so the product topology is contained within any other topology that satisfies the continuity of $\pi_j$ for each $j$.
\end{rem}

Not only are projections continuous, they are also open!

\begin{result}{theorem}
  Let $X \coloneqq X_1 \x \cdots \x X_n$ eb the product of topological spaces $X_1, \ldots, X_n$. Then each projection $\pi_j \colon X \to X_j$ is an open map.
\end{result}
\begin{pf}
  Let $U \coloneqq U_1 \x \cdots \x U_n$ be a basic open set. Then $\pi_j(U) = U_j$ is open. Since maps preserve unions, it follows that the image of any open subset of $X$ under $\pi_j$ is open in $X_j$.
\end{pf}

I can't think of a way to introduce this coming theorem yet lmao.
\begin{result}{theorem}
  Let $E$ be a topological space and let $f$ be a map from $E$ to $X \coloneqq X_1 \x \cdots \x X_n$. Then $f$ is continuous if and only if each $\pi_j \circ f \colon E \to X_j$ is continuous for $1 \leq j \leq n$.
\end{result}
\begin{pf}
  ($\Rightarrow$) Since $f$ and $\pi_j$ are continuous, the composition $\pi_j \circ f$ is continuous.

  ($\Leftarrow$) Suppose each $\pi_j \circ f$ is continuous and let $U \coloneqq U_1 \x\cdots\x U_n$ be a basic open set in $X$. Then
    \begin{align*}
      f^{-1}(U) &= f^{-1}(\pi_1^{-1} (U_1) \cap \cdots \cap \pi_n^{-1}(U_n)) \\
        &= (f^{-1} \circ \pi_1^{-1})(U_1) \cap \cdots \cap (f^{-1}\circ \pi_n^{-1})(U_n)) \\
        &= (\pi_1 \circ f)^{-1}(U_1) \cap \cdots \cap (\pi_n \circ f)^{-1}(U_n))
    \end{align*}
  is a finite intersection of open sets and hence is open. Moreover, inverse preserve unions, so, similarly, the preimage of any open set under $f$ is open. So $f$ is continuous.

  This completes the proof.
\end{pf}










\subsection{Quotient spaces}
\textbf{FINISH}










\subsection{Separation axioms}

Convergence of sequences depends on the topology. For metric spaces, we could discuss limits by invoking the metric topology. In general, the picture is not so clear.

\begin{ex}
  In $\Z$ with the cofinite topology, the sequence $(1,2,3,\ldots)$ converges to every point in $\Z$.
\end{ex}

We control this with separation axioms. [I personally like to think of separation axioms is how much you can `separate' or distinguish subsets of a topological space.]

\begin{defn}
  A topological space $X$ is $T_0$ if and only if its points are ``topologically distinguishable." That is, $X$ is $T_0$ if and only if for all $x,y \in X$ with $x \neq y$, there exists an open set containg $x$ and not $y$ or containing $y$ and not $x$.
\end{defn}

\begin{ex}
  Take a set $X$ of more than one point with the indiscrete topology. Then, $X$ is not $T_0$.
\end{ex}










\subsection{Compactness}

Compactness is the same thing from metric space topology (the open cover definition, I mean).

\begin{defn}[Compactness]
  Let $X$ be a topological space. A subset $Y \subseteq X$ is \deft{compact} if and only if for every open cover $\set{U_i}_{i \in I}$ of $Y$, there is a finite set $\set{i_1, \ldots, i_n}$ such that $Y \subseteq U_{i_1} \cup \cdots \cup U_{i_n}$.
\end{defn}
\begin{slogan}
  Compact $\iff$ every open cover has a finite subcover.
\end{slogan}

Now, recall the following proposition:
\textbf{[insert proposition about closed subset of compact is compact]}

\begin{rem}
  We had a converse to this proposition (i.e., if $V \subseteq X$ is compact, then $V$ is closed) for metric spaces, but we don't have this for general topological spaces. However, it suffices to have our space be Hausdorff for this converse to hold.
\end{rem}

\begin{result}{proposition}
  Suppose that $X$ is a Hausdorff topological space and that $K$ is a compact subset of $X$, then $K$ is closed.
\end{result}
\begin{pf}
  We will show that $X \sm K$ is open. Let $x \in X \sm K$. Since $X$ is Hausdorff, for each $k \in K$, there exists $U_k \ni k$ and $V_k \ni x$ with $U_k$ and $V_k$ disjoint and open. Now $K \subseteq \bigcup_{k \in K}^{} U_k = \colon U$ and since $K$ is compact, there exists finitely many $k_1, \ldots, k_n$ such that $K \subseteq U_{k_1} \cup \cdots \cup U_{k_n}$. Define
    \[
      V \coloneqq V_{k_1} \cap \cdots \cap V_{k_n}.
    \]
  Then $x \in V$ and $V \cap U$ is empty, which implies that $V \cap K$ is empty and $V \subseteq X \sm K$. Thus $X \sm K$ is open and $K$ is closed.
\end{pf}

\begin{result}{corollary}
  A compact Hausdorff space is $T_3$.
\end{result}
\begin{pf}
  \textbf{FINISH}
\end{pf}

\begin{result}{proposition}
  A compact Hausdorff topological space is $T_4$.
\end{result}
\begin{pf}
  asdf
\end{pf}







\end{document}
