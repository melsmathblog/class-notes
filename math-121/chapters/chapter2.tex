\documentclass[class=article, crop=false]{standalone}
\input{preamble}

\externaldocument{chapter1}

\begin{document}

\section{Basics of topology}

\subsection{Topological spaces}

We start with the titular definition.
\begin{defn}[Topology]
  A \deft{topology} on a set $X$ is a collection $\mathcal T$ of subsets of $X$ such that
    \begin{enumerate}[(1)]
      \item $\es, X \in \mathcal T$;
      \item any union of sets in $\mathcal T$ is in $\mathcal T$; and
      \item any finite intersection of sets in $\mathcal T$ is in $\mathcal T$.
    \end{enumerate}
  The elements of $\mathcal T$ are called \deft{open sets}. A topological space is a pair $(X,\mathcal T)$.
\end{defn}
\begin{rem}
  A topology is a prescription for which sets are open.
\end{rem}

Let's relate this back to something more familiar and intuitive: open sets in metric spaces. Recall the following properties of open sets in metric spaces: the empty set and the whole space are open; any union of open sets is open; and any finite intersection of open sets is open. I think of our definition of topology as a generalization of that.

Now let's get to some examples of topologies. (Some are gross, which seems to be a common theme in topology. Things can get pretty pathological at times.)

\begin{ex} \ldr
  \begin{enumerate}[(1)]
    \item If $X$ is a metric space, the collection of open sets is a topology called the \deft{metric topology}.

    \item If $X$ is any set, the family $\mathcal T \coloneqq \set{\es, X}$ forms a topology called the \deft{indiscrete topology}.

    \item If $X$ is any set, the family $\mathcal T \coloneqq \mathcal P (X)$ (the collection of all subsets) forms the \deft{discrete topology}.

    \item Let $X \coloneqq \set{a,b}$. Then $\mathcal T \coloneqq \set{\es, \set a, X}$ is a topology.
  \end{enumerate}
\end{ex}

Now let's get to closed sets. (They're what you think they are, if you're familiar with metric topology.)

\begin{defn}[Closed set]
  Let $X$ be a topological space. A subset $C$ of $X$ is \deft{closed} if and only if $X\sm C$ is open.
\end{defn}
\noindent It turns out that our aherent points thing from metric spaces still works but this is the more topology-core definition of closed. (We'll see this more and more in topology as we continue to study it, but things are usually defined in terms of open sets.)

Let's keep translating some of our basic metric-space-type ideas to mroe general topology. Next we have the neighborhood (another name for an open ball in metric spaces).

\begin{defn}[Neighborhood]
  Let $X$ be a topological space. An open \deft{neighborhood} of $x \in X$ is an open set $U$ with $x \in U$.
\end{defn}

Recall that we would define open sets in metric spaces using interior points. We actually have a very similar idea in general topological spaces using open neighborhoods instead of the open ball.

\begin{defn}
  Let $X$ be a topological space and suppose $S$ is a subset of $X$. A point $x \in X$ is an \deft{interior point} of $S$ if and only if there exists an open neighborhood $U$ of $x$ that is contained in $S$

  The \deft{interior} of $S$ is the set of interior points of $S$ and is denoted $S^\circ$.
\end{defn}

\begin{prop}
  Let $X$ be a topological space and let $S$ be a subset of $X$. Then
  \begin{enumerate}[(1)]
    \item $S^\circ$ is open and
    \item $S$ is open if and only if $S = S^\circ$.
  \end{enumerate}
\end{prop}
\begin{pf}
  ad
\end{pf}



\end{document}
