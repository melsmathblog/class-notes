% Document set up
\usepackage[margin=1in]{geometry}

% Packages
\usepackage[utf8]{inputenc}
\usepackage[skins]{tcolorbox}
\usepackage{graphicx}
\usepackage{latexsym,amsfonts,amssymb,amsthm,amsmath,mathtools}
\usepackage[shortlabels]{enumitem}
\setlist[enumerate]{listparindent=24pt}
\usepackage{empheq}
\usepackage{tikz-cd}
\usepackage{ytableau}
\usepackage{zref}% http://ctan.org/pkg/zref
\usepackage{thmtools}
\usepackage{thm-restate}
% \usepackage{cleveref}
\usepackage{mathrsfs}
\usepackage{extarrows}
\usepackage{fancyhdr}
\usepackage{import}
\usepackage{pdfpages}
\usepackage{standalone}
\usepackage{xr}

% Gives subsubsubsection
\usepackage{titlesec}
\usepackage{hyperref}

\titleclass{\subsubsubsection}{straight}[\subsection]

\newcounter{subsubsubsection}[subsubsection]
\renewcommand\thesubsubsubsection{\thesubsubsection.\arabic{subsubsubsection}}
\renewcommand\theparagraph{\thesubsubsubsection.\arabic{paragraph}} % optional; useful if paragraphs are to be numbered

\titleformat{\subsubsubsection}
{\normalfont\normalsize\bfseries}{\thesubsubsubsection}{1em}{}
\titlespacing*{\subsubsubsection}
{0pt}{3.25ex plus 1ex minus .2ex}{1.5ex plus .2ex}

\makeatletter
\renewcommand\paragraph{\@startsection{paragraph}{5}{\z@}%
{3.25ex \@plus1ex \@minus.2ex}%
{-1em}%
{\normalfont\normalsize\bfseries}}
\renewcommand\subparagraph{\@startsection{subparagraph}{6}{\parindent}%
{3.25ex \@plus1ex \@minus .2ex}%
{-1em}%
{\normalfont\normalsize\bfseries}}
\def\toclevel@subsubsubsection{4}
\def\toclevel@paragraph{5}
\def\toclevel@paragraph{6}
\def\l@subsubsubsection{\@dottedtocline{4}{7em}{4em}}
\def\l@paragraph{\@dottedtocline{5}{10em}{5em}}
\def\l@subparagraph{\@dottedtocline{6}{14em}{6em}}
\makeatother

\setcounter{secnumdepth}{4}
\setcounter{tocdepth}{4}


%   Proof symbol
\renewcommand{\qedsymbol}{\ensuremath{\blacksquare}}

%   File path command
\newcommand{\filepath}{\import{C:/Users/Loopi/github/class-notes/math-131abh/chapters/figures}}

%   Renew commands
\renewcommand{\vec}[1]{\mathbf{#1}}

% code snippet?
\definecolor{codegray}{gray}{0.9}
\newcommand{\code}[1]{\colorbox{codegray}{\texttt{#1}}}
% \renewcommand{\emph}{\code}

%   Math operators
\DeclareMathOperator{\lcm}{lcm} % lcm
\DeclareMathOperator{\im}{im} % image
\DeclareMathOperator{\Stab}{Stab} % stabilizer of an element
\DeclareMathOperator{\Orb}{Orb} % orbit of an element
\DeclareMathOperator{\pset}{\mathcal P} % power set
\DeclareMathOperator{\diam}{diam} % diameter of a set
\DeclareMathOperator{\coker}{coker} % cokernel

%   Shorthand
\newcommand{\closure}[1]{\overline{#1}} % Closure notation // Legacy command
\newcommand{\Syl}[1]{\ensuremath{\mathrm{Syl}_{#1}}}
\newcommand{\clos}[1]{\overline{#1}} % Closure notation
\newcommand{\ceil}[1]{\left \lceil #1 \right \rceil} % ceiling function
\newcommand{\floor}[1]{\left \lfloor #1 \right \rfloor} % floor function
\newcommand{\seq}[1]{\left( #1 \right)} % sequence (131AH)
\newcommand{\set}[1]{\ensuremath{\left\{ #1 \right\}}} % set
\newcommand{\nrm}{\ensuremath{\triangleleft}} % normal subgroup
\newcommand{\nrmeq}{\ensuremath{\trianglelefteq}} % normal subgroup
\newcommand{\nrmneq}{\ensuremath{\triangleleftneq}} % normal proper subgroup
\newcommand{\cyc}[1]{\left \langle #1 \right \rangle} % cyclic group
\newcommand{\R}{\ensuremath{\mathbb{R}}} % Real numbers
\newcommand{\Q}{\ensuremath{\mathbb{Q}}} % Rational Numbers
\newcommand{\I}{\ensuremath{\mathbb{I}}} % Irrational numbers
\newcommand{\Z}{\ensuremath{\mathbb{Z}}} % Integers
\newcommand{\F}{\ensuremath{\mathbb{F}}} % Fields
\newcommand{\N}{\ensuremath{\mathbb{N}}} % Naturals
\newcommand{\C}{\ensuremath{\mathbb{C}}} % Complex numbers
\newcommand{\es}{\ensuremath{\varnothing}} % empty set
\newcommand{\sm}{\ensuremath{\setminus}} % better set minus
\newcommand{\nss}{\not \subset} % not subset
\newcommand{\eps}{\ensuremath{\epsilon}} % epsilon (legacy)
\newcommand{\e}{\ensuremath{\epsilon}} % epsilon
\newcommand{\x}{\ensuremath{\times}} % cartesian product
\newcommand{\lam}{\ensuremath{\lambda}} % lambda
\newcommand{\Lam}{\ensuremath{\Lambda}} % Lambda
\newcommand{\melon}{\ensuremath{\ni\!\!\!\!\!\!\in}} % melon
\newcommand{\bC}[2]{
\begin{tikzpicture}
  \draw plot [smooth cycle, tension=.5] coordinates {(#1*-0.106,#1*.12) (#1*-0.196,#1*0.266) (#1*-0.258,#1*0.44) (#1*-0.278,#1*0.535) (#1*-0.284,#1*0.635) (#1*-0.272,#1*0.73) (#1*-0.245,#1*0.825) (#1*-0.206, #1*0.892) (#1*-0.136,#1*0.958) (#1*-0.052,#1*0.993) (#1*0.057,#1*0.99) (#1*0.173,#1*0.944) (#1*0.27,#1*0.854) (#1*0.36,#1*0.72) (#1*0.45,#1*0.557) (#1*0.58, #1*0.44) (#1*0.738, #1*0.353) (#1*0.87, #1*0.253) (#1*0.972, #1*0.113) (#1*0.996,#1*-0.036) (#1*0.918, #1*-0.18) (#1*0.776, #1*-0.256) (#1*0.627,#1*-0.28) (#1*0.446, #1*-0.258) (#1*0.28, #1*-0.201) (#1*0.134, #1*-0.117) (0,0)};

  #2
\end{tikzpicture}
}

% gives fancy E
\DeclareFontFamily{U}{calligra}{}
\DeclareFontShape{U}{calligra}{m}{n}{<->callig15}{}

\newcommand{\E}{{\!\!\text{\usefont{U}{calligra}{m}{n}E}\,\,}}

% Colors
\definecolor{theoremBackground}{RGB}{212,232,246}
\definecolor{theoremTitle}{RGB}{52,114,163}
\definecolor{definitionBackground}{RGB}{255,248,197}
\definecolor{definitionTitle}{RGB}{182,93,27}
\definecolor{exampleTitle}{RGB}{3, 192, 60}
\definecolor{exampleBackground}{RGB}{209,255,227}
\definecolor{problemTitle}{RGB}{0,0,0}
\definecolor{problemBackground}{RGB}{255,225,225}
\definecolor{topicTitle}{RGB}{0,0,0}
\definecolor{topicBackground}{RGB}{239, 146, 122} % fix later

% Basic environments
\newtheorem{dummy}{***}[section] % Used so that theorems, definitions, etc can have same counter within subsections
\newtheorem{theorem}[dummy]{\color{theoremTitle}Theorem}
\newtheorem{lemma}[dummy]{\color{theoremTitle}Lemma}
\newtheorem{factBasic}[dummy]{\color{theoremTitle}Fact}
\newtheorem{corollary}[dummy]{\color{theoremTitle}Corollary}
\newtheorem{proposition}[dummy]{\color{theoremTitle}Proposition}
\newtheorem{axiom}[dummy]{\color{theoremTitle}Axiom}


\theoremstyle{definition}
\newtheorem{notation}[dummy]{\color{definitionTitle}Notation}
\newtheorem{problem}{\color{problemTitle}Problem}
\newtheorem*{topicBasic}{\color{topicTitle}Topic(s)}
\newtheorem*{remark}{Remark}
\newtheorem{example}[dummy]{\color{exampleTitle}Example}
\newtheorem{definition}[dummy]{\color{definitionTitle}Definition}
\newtheorem*{sloganBasic}{\color{topicTitle}Slogan}

% Custom Environments

\newenvironment{ax} % axiom environment
{
\begin{tcolorbox}[colback=theoremBackground, colframe=theoremBackground, sharp corners, before upper={\parindent24pt}]
  \begin{axiom}
    }
    {
  \end{axiom}
\end{tcolorbox}
}

\newenvironment{rem} % simple remark environment
{
\begin{remark}
  }
  {
\end{remark}
}

\newenvironment{prop} % colored Proposition environment
{
\begin{tcolorbox}[colback=theoremBackground, colframe=theoremBackground, sharp corners, before upper={\parindent24pt}]
  \begin{proposition}
    }
    {
  \end{proposition}
\end{tcolorbox}
}


\newenvironment{thm} % Colored theorem environment
{
\begin{tcolorbox}[colback=theoremBackground, colframe=theoremBackground, sharp corners, before upper={\parindent24pt}]
  \begin{theorem}
    }
    {
  \end{theorem}
\end{tcolorbox}
}


\newenvironment{defn} % Colored definition environment
{
\begin{tcolorbox}[colback=definitionBackground, colframe=definitionBackground, sharp corners, before upper={\parindent24pt}]
  \begin{definition}
    }
    {
  \end{definition}
\end{tcolorbox}
}

\newenvironment{lem} % Colored lemma environment
{
\begin{tcolorbox}[colback=theoremBackground, colframe=theoremBackground, sharp corners, before upper={\parindent24pt}]
  \begin{lemma}
    }
    {
  \end{lemma}
\end{tcolorbox}
}

\newenvironment{fact} % Colored fact environment
{
\begin{tcolorbox}[colback=theoremBackground, colframe=theoremBackground, sharp corners, before upper={\parindent24pt}]
  \begin{factBasic}
    }
    {
  \end{factBasic}
\end{tcolorbox}
}

\newenvironment{coro} % Colored corollary environment
{
\begin{tcolorbox}[colback=theoremBackground, colframe=theoremBackground, sharp corners, before upper={\parindent24pt}]
  \begin{corollary}
    }
    {
  \end{corollary}
\end{tcolorbox}
}

\newenvironment{ex} % Colored example environment
{
\begin{tcolorbox}[colback=exampleBackground, colframe=exampleBackground, sharp corners, before upper={\parindent24pt}]
  \begin{example}
    }
    {
  \end{example}
\end{tcolorbox}
}

\newenvironment{nota} % Colored notation environment
{
\begin{tcolorbox}[colback=definitionBackground, colframe=definitionBackground, sharp corners, before upper={\parindent24pt}]
  \begin{notation}
    }
    {
  \end{notation}
\end{tcolorbox}
}

\newenvironment{prob}
{
\stepcounter{section}
\begin{tcolorbox}[colback=problemBackground, colframe=problemBackground, sharp corners, before upper={\parindent24pt}]
  \begin{problem}
    }
    {
  \end{problem}
\end{tcolorbox}
}

\newenvironment{pf}
{
\begin{proof}[\textbf{\emph{Proof}}]
  }
  {
\end{proof}
}

\newenvironment{topic}
{
\begin{tcolorbox}[colback=topicBackground, colframe=topicBackground, sharp corners, before upper={\parindent24pt}]
  \begin{topicBasic}
    }
    {
  \end{topicBasic}
\end{tcolorbox}
}

\newenvironment{slogan}
{
\begin{tcolorbox}[colback=topicBackground, colframe=topicBackground, sharp corners, before upper={\parindent24pt}]
  \begin{sloganBasic}
    }
    {
  \end{sloganBasic}
\end{tcolorbox}
}

\raggedbottom
