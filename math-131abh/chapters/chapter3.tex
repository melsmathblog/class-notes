\documentclass[class=article, crop=false]{standalone}
\input{preamble}

\externaldocument{chapter1}
\externaldocument{chapter2}

\begin{document}

\section{Series}

\subsection{Basics}

What does $\sum_{j=1}^{\infty} a_j = A$ (for some sequence $(a_n)$ and some number $A$) even mean? Well, in general, $\sum_{j=1}^{N} a_j = a_1 + \cdots + a_N$, so we're saying something like $a_1+ a_2 + a_3 + \cdots = A$. In particular, what we mean is the following: $\lim S_n = A$ where $S_n = \sum_{j=1}^{n} a_j$. So, because of this we can take a lot of my results from sequences and apply them to series!!! Super cool stuff right here! $:>$ But before we do that let's put we just said into an actual definition.
\begin{defn}
    Let $(a_n)$ be a sequence. Then,
        \[
            \sum_{j=m}^{n} a_j \coloneqq a_m + a_{m+1} + a_{m+2} + \cdots + a_n
        \]
    for $m \leq n$. With $(a_n)$ we associate the sequence $(S_n)$ where
        \[
            S_n \coloneqq \sum_{j=1}^{n} a_j.
        \]
    For $(S_n)$ we also use the symbolic expression
        \[
            a_1 + a_2 + a_3 + \cdots
        \]
    or
        \[
            \sum_{j=1}^{\infty} a_j. \tag{1}
        \]
    (1) is called an \textit{infinite series} or just \textit{series}. $(S_n)$ is called the \textit{sequence of partial sums of the series}. If $\lim S_n = S$, we say that the series converges and we write
        \[
            \sum_{j=1}^{\infty} a_j = S.
        \]
    $S$ is called the \textit{sum} of the series. If $(S_n)$ diverges, then the sum diverges.

    When the bounds are unambiguous we simply write $\sum a_j$.
\end{defn}

The following is a direct translation from sequence results to series.
\begin{thm}[Algebraic Limit Theorem for Series]
    If $\sum_{j=1}^{\infty} a_j = A$ and $\sum_{j=1}^{\infty} b_j = B$, then
        \begin{enumerate}[\normalfont(i)]
            \item $\sum_{j=1}^{\infty} ca_j = cA$ for all $c \in \R$;
            \item $\sum_{j=1}^{\infty} (a_j+b_j) = A+B$.
        \end{enumerate}
\end{thm}
\begin{rem}
    We won't talk about products of series just yet. ;)
\end{rem}

Remember Cauchy sequences? Well, because we can treat series like sequences (basically), it ends up being pretty improtant.

\begin{thm}{\label{thm:cauchycritforseries}}
    The series $\sum_{j=1}^{\infty} a_j$ converges if and only if for all $\e > 0$ there exists an integer $N$ such that whenever $n > m \geq N$ it follows that
        \[
            |a_{m+1}+a_{m+2}+a_{m+3}+\cdots+a_n| < \e.
        \]
\end{thm}
\begin{pf}
    Notice that the final line is equivalent to $|S_n-S_m| < \e$. Then also notice that the theorem is saying $(S_n)$ converges if and only if it is Cauchy, which is true by the completeness of $\R$.
\end{pf}

\begin{coro}
    If $\sum_{j=1}^{\infty} a_j$ converges, then $\lim a_n = 0$.
\end{coro}
\begin{pf}
    Consider the case of $n=m+1$ with regard to Theorem \ref{thm:cauchycritforseries}. We then get that $|a_n| < \e$ and $\lim a_n = 0$.
\end{pf}
However, is the converse of this corollary true? No!
\begin{ex}[Harmonic series]
    The Harmonic Series is $\sum_{n=1}^{\infty} 1/n$. $\lim (1/n) = 0$, but the sum does not converge! (We'll prove this soon.)
\end{ex}

\begin{ex}[Geometric Series]
    A geometric is a series of the form $\sum_{j=0}^{n} a r^j$ for some common ratio (number) $r$. A geometric series converges if and only if $|r| < 1$. We leave this as an exercise as the identity $\sum_{j=0}^{n} a r^j = \frac{a(1-r^n)}{1-r}$ should be enough to do this. Should be pretty straight forward.
\end{ex}




\subsection{Tests for convergence}

There are a ton of tests. Here are the main ones I know (not all were shown in class).

\begin{thm}[Cauchy Condensation Test]
    Suppose $(a_n)$ is decreasing and satisfies $a_n \geq 0$ for all $n$. Then, the series $\sum_{j=1}^{\infty} a_j$. Then the series $\sum_{j=1}^{\infty} a_j$ converges if and only if the series $\sum_{j=1}^{\infty} 2^ja_{2^j}$ converges.
\end{thm}
\begin{pf}
    \textbf{finish}
\end{pf}

\begin{thm}[Harmonic Series Test]
    The series $\sum_{n=1}^{\infty} 1/n^p$ converges if and only if $p > 1$.
\end{thm}
\begin{pf}
    \textbf{FINISH}
\end{pf}

\begin{thm}[Comparison Test]
    Suppose $(a_n)$ and $(b_n)$ satisfy $0 \leq a_n \leq b_n$ for all $n$. Then,
        \begin{enumerate}[\normalfont(i)]
            \item If $\sum_{j=1}^{\infty} b_j$ converges, then $\sum_{j=1}^{\infty} a_j$ converges;
            \item If $\sum_{j=1}^{\infty} a_j$ diverges, then $\sum_{j=1}^{\infty} b_j$ diverges.
        \end{enumerate}
\end{thm}
\begin{pf}
    Let $\e > 0$ be arbitrary.

    (i) Suppose $\sum_{j=1}^{\infty} b_j$ converges. From Theorem \ref{thm:cauchycritforseries}, there exists an integer $N$ such that $n > m \geq N$ such that $|b_{m+1}+\cdots+b_n| < \e$. Because $0 \leq a_n \leq b_n$ for all $n$, we have that
        \[
            |a_{m+1}+\cdots + a_n| \leq |b_{m+1}+\cdots+b_n| < \e.
        \]
    Thus $\sum_{j=1}^{\infty} a_j$ converges.

    (ii) Similar argument to the proof of (i).
\end{pf}

\begin{thm}[Absolute Convergence Test]
    If $\sum_{j=1}^{\infty} |a_j|$ converges, then $\sum_{j=1}^{\infty} a_j$ converges.
\end{thm}
\begin{pf}
    Let $\e > 0$ be arbitrary. It follows from Theorem \ref{thm:cauchycritforseries} that there exists an integer $N$ such that $n > m \geq N$ implies that $||a_{m+1}|+\cdots+|a_n|| = |a_{m+1}| + \cdots + |a_n|<\e$. It follows from Triangle Inequality that
    $|a_{m+1} + \cdots + a_n | \leq |a_{m+1}| + \cdots + |a_n|<\e$, so $\sum_{j=1}^{\infty} a_j$ converges.
\end{pf}

Converse is not true however! $\sum_{j=1}^{\infty} (-1)^j/j$ converges but $\sum_{j=1}^{\infty} 1/j$ does not. But how do we know that $\sum_{j=1}^{\infty} (-1)^j/j$ converges? With the following test:

\begin{thm}[Alternating Series Test]
    Let $(a_n)$ be a seuqence satisfying,
        \begin{enumerate}[\normalfont(i)]
            \item $a_1 \geq a_2 \geq a_3 \geq \cdots \geq a_n \geq a_{n+1} \geq \cdots$ and
            \item $\lim a_n = 0$.
        \end{enumerate}
    Then, the alternating series $\sum_{j=1}^{\infty} (-1)^ja_j$ converges.
\end{thm}
\begin{pf}
    \textbf{ FINISH}
\end{pf}


\end{document}
