\documentclass[class=article, crop=false]{standalone}
\input{preamble}

\externaldocument{chapter1}
\externaldocument{chapter2}
\externaldocument{chapter3}
\externaldocument{chapter4}
\externaldocument{chapter5}
\externaldocument{chapter6}
\externaldocument{chapter7}
\externaldocument{chapter8}

\begin{document}

\section{Equicontinuity and Arzela-Ascoli Theorem}

\subsection{Equicontinuity}

So we've covered continuity and uniformy continuity, but there's at least one more type: equicontinuity. Let's get right into the definition as I don't find the name itself to be particularly enlightening.

\begin{defn}[Equicontinuity]
    Let $(X,d_x)$ and $(Y,d_y)$ be metric spaces. A sequence of functions $(f_n)$ where each $f_n \colon X \to Y$, is called \textit{equicontinuous} if for every $\e > 0$ there exists a $\delta > 0$ such that $d_y(f_n(x),f_n(y)) < \e$ for all $n=1,2,3,\ldots$ and $d_x(x,y) < \delta$.
\end{defn}

In quantifiers this is: \textbf{get help on this. how do i translate purely into quantifiers?}

What does equicontinuity really mean, however? It wasn't initially clear to me, but the way I best understand it by comparing it with the other types of continuity we've learned about. In particular, by comparing what $\delta$ can depend on in our definitions.

\begin{enumerate}[$\bullet$]
    \item Continuity: $\delta$ may depend on $\e$ and the point being considered.
    \item Uniform continuity: $\delta$ may depend on $\e$ (and CANNOT depend on the point being considered)
    \item Equicontinuity: $\delta$ may depend on $\e$ and \textbf{FINISH}
\end{enumerate}


\end{document}
