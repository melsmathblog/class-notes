\documentclass[class=article, crop=false]{standalone}
\input{preamble}

\externaldocument{chapter1}
\externaldocument{chapter2}
\externaldocument{chapter3}
\externaldocument{chapter4}
\externaldocument{chapter5}

\begin{document}

\section{(Aside:) Completeness}

\subsection{Basics}

In the very first section (in the first paragraph, in fact!), we said the rationals are `incomplete.' We will now explain exactly what that meant.
\begin{defn}[Completeness]
    A metric space $(X,d)$ is complete if and only if every Cauchy sequence in $(X,d)$ converges in $(X,d)$.
\end{defn}

Do we have an example of a complete metric space? Indeed, we do!
\begin{ex}
    $\R$ with the Euclidean metric is complete.
\end{ex}
\begin{pf}
    This is Theorem \ref{thm:completeness of R}.
\end{pf}

In fact, we have many more examples of complete metric spaces! Say $\ell_2$ or $C([0,1])$ (with the sup-norm metric), or even $\R^n$ (with the Euclidean metric) for any finite $n$! (We'll prove $\ell_2$, $C([0,1])$, and $\R^2$ is complete in the upcoming section.) Now to some examples.



\subsection{Examples of cool complete metric spaces (with proof)}

\begin{fact}
    $\R^2$ with the Euclidean metric is complete.
\end{fact}
\begin{pf}
    Let $\e > 0$. Suppose $(\vec a_n)$ is a Cauchy sequence. We may rewrite $(\vec a_n)$ as $((x_n, y_n))$. First we will show that both $(x_n)$ and $(y_n)$ are Cauchy in $\R$.

    Because $(\vec a_n)$ is Cauchy, there exists an integer $N$ such that $d(\vec a_i, \vec a_j) = \sqrt{(x_i-x_j)^2+(y_i-y_j)^2} < \e$ if $i,j \geq N$. It follows that
        \begin{align*}
            |x_i-x_j| = \sqrt{(x_i-x_j)^2} \leq \sqrt{(x_i-x_j)^2+(y_i-y_j)^2} &< \e \\
            |y_i-y_j| = \sqrt{(y_i-y_j)^2} \leq \sqrt{(x_i-x_j)^2+(y_i-y_j)^2} &< \e
        \end{align*}
    if $i,j \geq N$. Thus $(x_n)$ and $(y_n)$ are Cauchy in $\R$ (with the Euclidean metric), so they have limits $x$ and $y$, respectively. We claim that $\lim \vec a_n = (x,y)$.

    Because $\lim x_n = x$, there exists an integer $N_1$ such that $n \geq N_1$ implies that $|x_n-x| < \e/\sqrt 2$. Similarly, there exists an $N_2$ such that $|y_n-y| < \e/\sqrt 2$ if $n \geq N_2$. Taking $n \geq \max \set{N_1, N_2}$, we get that
        \begin{align*}
            d(\vec a_n, \vec a) &= \sqrt{(x_n-x)^2 + (y_n-y)^2} \\
                &= \sqrt{|x_n-x|^2+|y_n-y|^2} \\
                &< \sqrt{ \left(\frac{\e}{\sqrt 2} \right)^2 + \left(\frac{\e}{\sqrt 2} \right)^2} \\
                &= \sqrt{\frac{\e^2}{2} + \frac{\e^2}{2}} \\
                &= \sqrt{\e^2} \\
                &= \e.
        \end{align*}
     Hence $\lim \vec a_n = (x,y)$ and $\R^2$ (with the Euclidean metric is complete.)
\end{pf}
\begin{coro}
    $\R^n$ with the Euclidean metric is complete.
\end{coro}
\begin{pf}
    Mimic the proof for the completeness of $\R^2$.
\end{pf}

\begin{fact}
    $\R^2$ with the Taxicab metric is complete.
\end{fact}
\begin{pf}
    Mimic the proof of the completeness of $\R^2$.
\end{pf}


\begin{fact}
    $\ell_2$ with the metric $d \colon \ell_2 \x \ell_2 \to \R$ defined by
        \[
            d((x_n),(y_n)) = \left( \sum_{n=1}^{\infty} |x_n-y_n|^2 \right)^{\frac{1}{2}}
        \]
    is complete.
\end{fact}

To make the following proof easier, we will denote the elements of $\ell_2$ in a new yet familiar way: as functions! So we will define $\ell_2$ as
    \[
        \ell_2 \coloneqq \set{f \colon \Z^+ \to \R \mid \sum_{k=1}^{\infty} |f(k)|^2 < \infty}.
    \]
Why can we do this? Well, sequences are just functions from the positive integers into the the reals, so writing them this way is perfectly fine. Now we may begin with the proof.
\begin{pf}
    das
\end{pf}



\begin{fact}
    $C([0,1])$ with the metric $d \colon C([0,1]) \x C([0,1]) \to \R$ defined by
        \[
            d(f,g) = \sup \set{|f(x)-g(x)| : x \in [0,1]}
        \]
    is complete.
\end{fact}
\begin{pf}
    Let $\e > 0$ be arbitrary. Let $(f_n)$ be a Cauchy sequence. Then there exists an integer $N$ such that $d(f_i,f_j) < \e/2$ if $i,j \geq N$. By the definition $d$ and supremum, we know that $|f_i(x)-f_j(x)| < \e/2 < \e$ and $(f_n(x))$ is a Cauchy sequence in $\R$ with the Euclidean metric for all $x \in [0,1]$.

    Taking $i,j \geq N$ again and then taking $j \to \infty$, we get
        \[
            \sup_{y \in [0,1]} |f_i(y)-f_j(y)| < \frac{\e}{2} \to \sup_{y \in [0,1]}|f_i(y)-f_0(y)| \leq \frac{\e}{2} < \e.
        \]
    Thus, $d(f_i,f_0) < \e$ if $i \geq N$ and $\lim f_n =f_0$.

    We must still show that $f_0 \in C([0,1])$ (i.e. we must show that $f_0$ is continuous). Let $c \in [0,1]$. Since $\lim f_n = f_0$, there exists an integer $N'$ such that for $|f_n(x)-f_0(x)| \leq d(f_n,f_0) < \e/3$ if $n \geq N'$ for all $x \in [0,1]$. Recall also that because $f_{N'}$ is continuous, there exists a $\delta > 0$ such that $|x-c| < \delta$
    implies that $|f_{N'}-f_{N'}(c)| < \e/3$. Now taking $|x-c| < \delta$, it follows that
        \[
            |f_0(x)-f_0(c)| \leq |f_0(x) - f_{N'}(x)| + |f_{N'}(x)-f_{N'}(c)| + |f_{N'}(c)-f_0(c)| < \frac{\e}{3}+\frac{\e}{3}+\frac{\e}{3} = \e.
        \]
    Thus $f_0$ is continuous.

    This completes the proof.
\end{pf}
\begin{rem}
    The final inequality in the proof is called the ``Three Term Estimate." It's very important stuff. Don't forget it! We might make a section/bit on it. idk
\end{rem}


\subsection{Complete subspaces}

When is the subset of a metric space complete? When it's closed!
\begin{fact}
    Let $(X,d)$ be a complete metric space and suppose $Y \subseteq X$. If $Y$ is closed in $X$, then $(Y, d|_{Y \x Y})$ is complete.
\end{fact}
\begin{pf}
    Suppose $(x_n)$ is a Cauchy sequence in $Y$. Then $(x_n)$ converges to $x$ in $X$. $Y$ contains all its limit points because it is closed, so $x \in Y$. Thus $Y$ is complete.
\end{pf}

Now back to topology!

\end{document}
