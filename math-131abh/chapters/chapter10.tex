\documentclass[class=article, crop=false]{standalone}
\input{preamble}

\externaldocument{chapter1}
\externaldocument{chapter2}
\externaldocument{chapter3}
\externaldocument{chapter4}
\externaldocument{chapter5}
\externaldocument{chapter6}
\externaldocument{chapter7}
\externaldocument{chapter8}
\externaldocument{chapter9}

\begin{document}

\section{Analysis on normed vector spaces}

This was/is rough, but very cool lmao. B)

\subsection{Vector spaces}

Nothing new here (except infinite dimensional vector spaces), but we are only considering vector spaces over $\R$.

\subsection{Norms}

Let's get right into the definition.

\begin{defn}[Norm]
    Let $V$ be a vector space. A \textit{norm} is a nonnegative function $f \colon V \to \R$ defined by the following:
        \begin{enumerate}[(1)]
            \item $f(\vec u) = 0$ if and only if $\vec u =\vec 0$,
            \item $f(a\vec u) = |a| f(\vec u)$,
            \item $f(\vec u + \vec v) \leq f(\vec u) + f(\vec v)$,
        \end{enumerate}
    where $\vec u, \vec v \in V$ and $a \in \R$.
\end{defn}
\begin{rem}
    We call (2) homogeneity and we call (3) being \textit{subadditive} or satisfying the triangle inequality. Most of the time, norms are also written $\| \cdot \|$ (where $\cdot$ represents an argument).
\end{rem}

I think of norms as sort of unary metrics (metrics take 1 argument) in a very rough sense. It gives us a rough sense of how `large' a vector is. Interestingly (and likely expectedly), we can define a metric using norms!

\begin{fact}
    Let $V$ be a vector space and let $\| \cdot \|$ be a norm on $V$. Then we can define a metric $d$ on $V \x V$ by the following:
        \[
            d(\vec u, \vec v) \coloneqq \|\vec u - \vec v \|.
        \]
\end{fact}
I won't prove this lol. It shouldn't be too rough though.

Now here are common norms you'll see in this section (probably):
\begin{ex}[Norms on Euclidean space]
    For this example, let $\vec x = (x_1, \ldots, x_n)$ be in $\R^n$.
    \begin{enumerate}[$\bullet$]
        \item The Euclidean norm (denoted by $\|\cdot\|_2$) on $\R^n$ is defined to be
            \[
                \| \vec x \|_2 \coloneqq \sqrt{\sum_{j=1}^{n} (x_j)^2)}
            \]
        is a norm on $\R^n$.

        \item The $L^1$ norm (denoted $\|\cdot\|_1$) on $\R^n$ is defined to be
            \[
                \| \vec x \|_1 \coloneqq \sum_{j=1}^{n} |x_j|
            \]
        is a norm on $\R^n$.

        \item The $\sup$ norm on (denoted $\|\cdot\|_\infty$) on $\R^n$ is defined to be
            \[
                \| \vec x\|_\infty \coloneqq \sup \set{|x_i| : i=1,\ldots,n}
            \]
        is a norm on $\R^n$.
    \end{enumerate}
\end{ex}

Now we can move on to a more interesting idea: the equivalence of norms!

\subsection{Equivalence of Norms}

\begin{defn}[Equivalent norms]
    Let $\| \cdot \|_0$ and $\| \cdot \|$ be norms on a vector space $V$. Then $\| \cdot \|_0$ and $\| \cdot \|$ are \textit{equivalent} if and only if there exists two positive real constants $a$ and $b$ such that for every $\vec v \in V$,
        \[
            a \| \vec v \|_0 \leq \| \vec v \| \leq b\| \vec v \|_0.
        \]
\end{defn}


Let's go through and show that some of our common norms are equivalent. Let's go for showing the $L^1$ and Euclidean norms.

\begin{fact}
    The $L^1$ norm and the Euclidean norm on $\R^n$ are equivalent.
\end{fact}
\begin{pf}

\end{pf}

\subsection{Equivalence of norms on finite dimensional vector spaces}

Is this a conditional? Is the quantifier version this: some is equicontinuous if and only if
\[
(\forall \epsilon > 0) (\exists \delta > 0) [y \in B(x;\delta) \land f \in F \implies |f(x)-f(y)| < \epsilon.
\]



\end{document}
